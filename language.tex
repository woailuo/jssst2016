\section{Language \(\mathcal{L}\)}\label{sec:language}
In this section we define an imperative language \(\mathcal{L}\) with
memory allocation and deallocation primitives, and for simplification
we only use pointers as values.
 
The syntax of the language \(\mathcal{L}\) is as follows.

\begin{eqnarray*}
  x,y,z,\dots \mbox{ (variables)} &\in& \VAR\\
  s \mbox{ (statements)} & ::= &  \SKIP \mid s_{1};s_{2} \mid *x \leftarrow y \mid \Free(x) \\
  & \mid & \LET x = \MALLOC \IN s \mid \LET x = \NULL\ \IN s  \\
  & \mid & \LET x = y \; \IN s \mid   \LET x = *y \; \IN s \\
  & \mid & \IFNULL(*x) \; \THEN s_{1}\; \ELSE s_{2} \mid f(\vec{x})\\
  & \mid & \scon\Sirx s \mid \Endconst  \\
  d \mbox{ (proc. defs.)} & ::= & \set{f \mapsto (x_1,\dots,x_n)s}\\
  D \mbox{ (definitions) } &::=& \langle d_1 \cup \dots \cup d_n \rangle\\
  P \mbox{ (programs)} &::=& \langle D, s \rangle\\
  E \mbox{ (context)} &::=& E;s \mid [ ] 
\end{eqnarray*}

\paragraph{Notation} \(\vec{x}\) is for a finite sequence \(\{x_1,...,x_n\}\),
where we assume that each element is distinct; \([\vec{x'}/\vec{x}]s\)
is for a term obtained by replacing each free occurrence of
\(\vec{x}\) in s with variables \(\vec{x'}\).\\


The \(\VAR\) is a countably infinite set of \emph{variables} and each
variable is a pointer. The statement \(\SKIP\) means "does nothing".
The statement \(s_1;s_2\) is a sequential execution of \(s_1\) and
\(s_2\). The statement \(*x \leftarrow y\) changes the content of cell
which is pointed to by \(x\) with the value \(y\). The statement
\(\Free(x)\) deallocates a memory cell which is pointed to by pointer
\(x\). The statement \(\LET x = e\ \IN s\) evaluates the expression
\(e\), binds \(x\) to the result, and executes \(s\). The expression
\(\Malloc()\) allocates a new memory cell. The expression \(\NULL\)
evaluates to the null pointer. The expression \(*y\) means
dereferencing a memory cell pointed to by \(y\). The statement
\(\IFNULL(*x) \THEN s_1 \ELSE s_2\) executes \(s_1\) if \(*x\) is
\(\NULL\) and executes \(s_2\) otherwise. The statement \(f(\vec{x})\)
expresses a procedure \(f\) with arguments \(\vec{x}\). The statement
\(\scon\Sirx s\) means \(\Sirx\) is a constant in statement \(s\). The
statement \(\Endconst\) means from this point \(\Sirx\) maybe not a
constant.

The \(d\) represents a procedure definition which maps a procedure
name \(f\) to its procedure body \((\vec{x})s\); The \(D\) represents
a set of procedure definitions \(\langle d_1 \cup\dots d_n \rangle\),
and each definition is distinct; The pair \(\langle D, s \rangle \)
represents a program, where \(D\) is a set of definitions and \(s\) is
a main statement; the \(E\) represents evaluation context.

\subsection{Operational semantics}
\label{sec:languageSemantics}
In this section we introduce operational semantics of language
\(\mathcal{L}\). We assume there is a countable infinite set
\(\mathcal{H}\) of \(\emph{heap addresses}\) ranged over by \(l\).

We use a configuration \(\langle H, R, s, n, C \rangle\) to
express a run-time state. Each elements in the configuration is as
follows.

\begin{itemize}
\item \(H\), a \emph{heap}, is a finite mapping from \(\mathcal{H}\)
  to \(\mathcal{H} \cup \set{\NULL}\);
\item \(R\), an \emph{environment}, is a finite mapping from \(\VAR\)
  to \(\mathcal{H} \cup \set{\NULL}\);
\item \(s\) is the statement that is being executed; 
\item \(n\) is a natural number that represents the number of memory
  cells available for allocation, which can be formalized to check
  memory leaks even for nonterminating programs;
\item \(C\) is a set related to current constant pointers, which
  contains \(\scon\Sirx\), \(\snull\) and \(\snnull\).
\end{itemize}

The operational semantics of the language \(\mathcal{L}\) is given by
a labeled transition relation \(\langle H, R, s, n, C \rangle
\xlongrightarrow{\rho}_D \langle H', R', s', n', C' \rangle\). The label
\(\rho\) is an action, which is as follows.
\begin{eqnarray*}
 \rho \mbox{ (label)} &::=& \Malloc(x') \mid \Free \mid \snull \mid \snnull \mid \tau %% \mid \sassx \mid \sassxn
 %% &&\mid \Startconst \mid \Endconst \mid \tau  \\
\end{eqnarray*}
The action \(\Malloc(x')\) expresses an allocation of a new memory
cell, and the new cell binds to a fresh variable \(x'\); \(\Free\)
expresses a deallocation of a memory cell; \(\snull\) means \(*x\) is
a null pointer, and \(\snnull\) not; \(\tau\) expresses the other
internal actions. For the operational semantics, we often omit
\(\tau\) in \(\xlongrightarrow{\tau}_D\).  The metavariable \(\sigma\)
is used for a finite sequence of actions \(\rho_1\dots\rho_n\). The
\(\xlongrightarrow{\rho_1\dots\rho_n}_D\) is short for
\(\xlongrightarrow{\rho_1}_D\xlongrightarrow{\rho_2}_D\dots\xlongrightarrow{\rho_n}_D\). The
\(\xLongrightarrow{\rho}_D\) means
\(\xlongrightarrow{}_D^*\xlongrightarrow{\rho}_D\xlongrightarrow{}_D^*\).
We write \(\xLongrightarrow{\rho_1\dots\rho_n}_D\) for
\(\xLongrightarrow{\rho_1}_D\dots\xLongrightarrow{\rho_n}_D\).

\paragraph{Notation} the \(\DOM(f)\) is a
mapping from function name \(f\) to its domain; for a map \(f\), the
\(f \set{x \mapsto v}\) and \( f \bs x\) are defined as follows:
\[
\begin{array}{rcl}
f \set{x \mapsto v} (w) &=&
\left\{
\begin{array}{ll}
v & \mbox{if \(x = w\)}\\
f(w) & \mbox{otherwise.}
\end{array}
\right.\\
(f \bs x)(w) &=&
\left\{
\begin{array}{ll}
\mbox{undefined} & \mbox{if \(x = w\)}\\
f(w) & \mbox{otherwise.}
\end{array}
\right.
\end{array}
\]
and \(filter\_C(C, *x)\) is defined by a pseudcode as follows:
\[
\begin{array}{lcl}
  filter\_C(C, *x) &=& let \; C' = C - \scon\Sirx \; in \\
  & & if \; \scon\Sirx \in \; C'\; then \; return \; C'\\
  & & else \; return \; C'\backslash\{\snull,\snnull \}
\end{array}
\]


Figure~\ref{fig:transitionRules} depicts the relation
\(\xlongrightarrow{\rho}_D\). Several important rules are listed as
follows.

\begin{itemize}
\item \rn{Sem-ConstSeq}: \(\scon\Sirx\) and \(\Endconst\) together
  guarantees a pointer pointed to by \(*x\) cannot be changed in the
  statement \(s\). The set \(C\) with the new added \(\scon\Sirx\)
  describes this status.

\item \rn{Sem-IfNullT} and \rn{Sem-IfConstNullT}: these two rules
  represents if \(\Sirx\) is a null pointer, the statement \(s_1\)
  will be executed. the difference of the two is if the \(\scon\Sirx\)
  is in set \(C\) then\(\snull\) is added to \(C\), which means
  \(\Sirx\) is a null pointer and cannot be updated from now on;
  otherwise \(\Sirx\) can be changed, like \rn{Sem-IfNullT}.

\item \rn{Sem-Free}: deallocating one memory cell pointed by \(x\) is
  to remove linkage of pointer variable \(x\) to heap; this action
  will release one memory cell space, which increments the number of
  available memory cells \(n\) by one. 

\item \rn{Sem-Malloc} and \rn{Sem-OutOfMem}: allocating one memory
  cell is described as updating the heap by adding a fresh heap
  variable \(l\) to anywhere \(v\) of the heap and adding the linkage
  of a fresh register variable \(x'\) to that \(l\); This action is
  allowed only if the number of available memory cells is positive;
  otherwise \(\OVERFLOW\).
  
\item \rn{Sem-AssignExn}, \rn{Sem-FreeExn} and \rn{Sem-DerefExn}: these
  rules express that accessing a null pointer or a dangling pointer
  will give raise to an exceptional state \(\MEMEX\).  However, in
  this paper we do not see the state \(\MEMEX\) is an erroneous state,
  hence a well-typed program may lead to these states. One thing we
  should notice the command \(\FREE\), if \(x\) is a null pointer,
  raises state \(\MEMEX\) in the current semantics, although it is
  equivalent to \(\SKIP\) in the C language.
  
\item \rn{Sem-AssignConstExn}: expressing that if a constant memory
  cell pointed to by \(x\) or its aliases are changed it will raise
  exceptional state \(\CONSTEX\).

\end{itemize}

In order to deal with a path-sensitive program to guarantee
\emph{total} memory-leak freedom, we redefined the several definitions
as follows.  defined as follows:
\begin{myDef}[total memory-leak freedom]
\label{df:ml}
A program \(\langle D, s \rangle\) is \emph{totally memory-leak free}
if there is a natural number \(n\) such that it does not require more
than \(n\) cells.
\end{myDef}
\begin{myDef}[Memory leak]
\label{df:ml}
A configuration \(\langle H, R, s, n, C \rangle\) \emph{goes overflow} if
there is \(\sigma\) such that \(\langle H, R, s, n, C \rangle
\xLongrightarrow{\sigma} \OVERFLOW\).  A program \(\langle D, s
\rangle\) \emph{consumes at least \(n\) cells} if \(\langle \emptyset,
\emptyset, s, n, \emptyset \rangle\) goes overflow.  
\end{myDef}


\begin{figure}
  \begin{minipage}{\textwidth}

 \infrule[Sem-ConstSkip]
   {C' = filter\_C(C, *x)}
   {\langle H, R,
     \Endconst, n, C \rangle \rightarrow_{D} \langle
     H, R, \SKIP , n, C' \rangle}

\vspace{2mm}

\infax[Sem-ConstSeq]
{\langle H, R, \scon\Sirx s, n, C \rangle
\rightarrow_{D}
\langle H, R, s;\Endconst , n, C \cup \{\scon\Sirx\} \rangle}

\vspace{2mm}  

\infax[Sem-Skip]
{\langle H, R, \SKIP;s, n, C \rangle
\longrightarrow_{D}
\langle H, R, s, n, C \rangle}

\vspace{2mm}

\infrule[Sem-Seq]
{\langle H, R, s_1, n, C  \rangle \xlongrightarrow{\rho}_{D} \langle H', R', s_1', n', C' \rangle}
{\langle H, R, s_1;s_2, n, C \rangle \xlongrightarrow{\rho}_{D} \langle H', R', s_1';s_2, n', C' \rangle}
 
\vspace{2mm}

\infrule[Sem-LetNull]
{x' \notin \DOM(R)}
{\langle H\coma R\coma  \LET x = \NULL \ \IN s , n, C \rangle
  \longrightarrow_{D}
  \langle H\coma R\Lfc x' \mapsto \NULL \Rfc \coma   \Lb x'/x \Rb s , n, C  \rangle }

\vspace{2mm}

\infrule[Sem-LetEq]
{x' \notin \DOM(R)}
{\langle H\coma R\coma \LET x = y \; \IN s , n, C  \rangle
  \longrightarrow_{D}
  \langle H\coma R\Lfc x' \mapsto R(y) \Rfc \coma   \Lb x'/x \Rb s , n, C\rangle }

\vspace{2mm}

\infrule[Sem-IfNullT]
{H(R(x)) = \NULL, \scon\Sirx \notin C}
{\langle H \coma R \coma \IFNULL\Sirx \ \THEN   s_{1}\ \ELSE\  s_{2} \coma  n, C \rangle
  \xlongrightarrow{\snull}_D
  \langle H\coma R\coma s_{1} \coma n, C \rangle}

\vspace{2mm}

\infrule[Sem-IfNullF]
{H(R(x)) \neq \NULL, \scon\Sirx \notin C}
{\langle H \coma R \coma \IFNULL\Sirx\ \THEN  s_{1}\ \ELSE  s_{2} \coma  n, C\rangle
  \xlongrightarrow{\snnull}_D
  \langle H\coma R\coma s_{2} \coma  n, C\rangle}


\vspace{2mm}

\infrule[Sem-IfConstNullT]
{H(R(x)) = \NULL, \scon\Sirx \in C }
{\langle H \coma R \coma \IFNULL\Sirx \ \THEN   s_{1}\ \ELSE\  s_{2} \coma  n, C \rangle
  \xlongrightarrow{\snull}_D
  \langle H\coma R\coma s_{1} \coma n, C\cup \{\snull\} \rangle}

\vspace{2mm}

\infrule[Sem-IfConstNullF]
{H(R(x)) \neq \NULL, \scon\Sirx \in C }
{\langle H \coma R \coma \IFNULL\Sirx\ \THEN  s_{1}\ \ELSE  s_{2} \coma  n, C\rangle
  \xlongrightarrow{\snnull}_D
  \langle H\coma R\coma s_{2} \coma  n, C\cup \{\snnull\} \rangle}

\vspace{2mm}

\infrule[Sem-Assign]
{\scon\Sirx \notin C}
{ \langle H \set{R(x) \mapsto v}, R, *x \leftarrow y , n, C\rangle \xlongrightarrow{}_{D}
  \langle H \Lfc R(x) \mapsto R(y) \Rfc , R, \SKIP , n, C \rangle }

\vspace{2mm}

\infrule[Sem-LetDeref]
{x' \notin \DOM(R) \andalso R(y) \in \DOM(H)}
{\langle H\coma R\coma  \LET x = *y \; \IN s , n, C \rangle
  \longrightarrow_{D}
  \langle H\coma R\Lfc x' \mapsto H(R(y)) \Rfc \coma   \Lb x'/x \Rb s , n, C\rangle }

\vspace{2mm}

\infrule[Sem-Free]
{R(x) \neq \NULL \mbox{ and } R(x) \in \DOM(H)}
{\langle H\set{R(x) \mapsto v}\coma R\coma \Free(x) , n, C \rangle \xlongrightarrow{\Free}_{D}
  \langle H\backslash R(x) \coma R \coma \SKIP , n+1, C \rangle}

\vspace{2mm}

\infrule[Sem-Malloc]
{l \notin \DOM(H) \andalso n > 0}
{\langle H\coma R\coma  \LET x = \Malloc() \; \IN s , n, C\rangle
  \xlongrightarrow{\Malloc{(x')}}_{D}
  \langle H \Lfc l \mapsto v\Rfc \coma R\Lfc x' \mapsto l \Rfc \coma   \Lb x'/x \Rb s , n-1, C  \rangle }

\vspace{2mm}
\begin{minipage}{0.5\textwidth}
\infrule[Sem-Call]
{D(f) = (\vec{y})s}
{ \langle H\coma R\coma  f(\vec{x}) , n, C\rangle
  \longrightarrow_{D}
  \langle H\coma R\coma  \Lb \vec{x}/\vec{y} \Rb s , n, C \rangle}
\end{minipage}
\begin{minipage}{0.5\textwidth}
\infrule[Sem-FreeExn]
{R(x) = \NULL \mbox{ or } R(x) \notin \DOM(H)}
{\langle H\coma R\coma \Free(x) , n, C\rangle \xlongrightarrow{\Free}_{D} \MEMEX}
\end{minipage}
\vspace{2mm}

\begin{minipage}{0.5\textwidth}
\infrule[Sem-AssignExn]
{R(x) = \NULL \mbox{ or } R(x) \notin \DOM(H)}
{\langle H\coma R\coma  *x \leftarrow y , n, C\rangle
  \longrightarrow_{D} \MEMEX }
\end{minipage}
\begin{minipage}{0.5\textwidth}
\infrule[Sem-DerefExn]
{R(y) = \NULL \mbox{ or } R(y) \notin \DOM(H)}
{\langle H\coma R\coma  \LET x = *y \; \IN s, n, C\rangle
    \longrightarrow_{D} \MEMEX}
\end{minipage}
\vspace{2mm}

\infrule[Sem-AssignConstExn]
{ \forall z.\scon(*z) \in C \mbox{ and } R(x) = R(z)}
{ \langle H \set{R(x) \mapsto v}, R, *x \leftarrow y , n, C\rangle \xlongrightarrow{}_{D}
  \CONSTEX}

%% \infrule[Sem-ConstExn]
%%         { H(R(x)) \neq H'(R'(x))
%%           \Rtab
%%           \langle H, R, s, n, C \rangle \xlongrightarrow{\rho}_{D} \langle H', R', s', n', C \rangle}
%% {\langle H, R, \scon\Sirx s, n, C \rangle \xlongrightarrow{\rho}_{D} \CONSTEX }

\infax[Sem-OutOfMem]
{ \langle H\coma R\coma \LET x = \Malloc() \ \IN s ,  0, C  \rangle    \xlongrightarrow{\Malloc{(x')}}_{D}
  \OVERFLOW}

\end{minipage}

\caption{Operational semantics of \(\mathcal{L}\).}
\label{fig:transitionRules}
\end{figure}
