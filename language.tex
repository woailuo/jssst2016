\section{Language \(\mathcal{L}\)}\label{sec:language}

This section defines an imperative language \(\mathcal{L}\) with memory
allocation and deallocation primitives.  

In the rest of this paper, we write \(\vec{x}\) for a finite sequence
\(x_1,\dots,x_n\); we assume that each element is distinct.  We write
\([\vec{x'}/\vec{x}]s\) for the term obtained by replacing each free
occurrence of \(\vec{x}\) in $s$ with variables \(\vec{x'}\).  We write
\(\DOM(f)\) for the partial function $f$.  We write \(f \set{x \mapsto
v}\) and \(f \bs x\) as follows.
\[
\begin{array}{rcl}
f \set{x \mapsto v} (w) &=&
\left\{
\begin{array}{ll}
v & \mbox{if \(x = w\)}\\
f(w) & \mbox{otherwise.}
\end{array}
\right.\\
(f \bs x)(w) &=&
\left\{
\begin{array}{ll}
\mbox{undefined} & \mbox{if \(x = w\)}\\
f(w) & \mbox{otherwise.}
\end{array}
\right.
\end{array}
\]


\subsection{Syntax}

\begin{figure*}[t]
 \begin{eqnarray*}
  x,y,z,\dots \mbox{ (variables)} &\in& \VAR\\
  s \mbox{ (statements)} & ::= &  \SKIP \mid s_{1};s_{2} \mid *x \leftarrow y \mid \Free(x) \\
  & \mid & \LET x = \MALLOC \IN s \mid \LET x = \NULL\ \IN s  \\
  & \mid & \LET x = y \; \IN s \mid   \LET x = *y \; \IN s \\
  & \mid & \IFNULL(*x) \; \THEN s_{1}\; \ELSE s_{2} \mid f(\vec{x})\\
  & \mid & \scon\Sirx s \mid \Endconst  \\
  d \mbox{ (proc. defs.)} & ::= & \set{f \mapsto (x_1,\dots,x_n)s}\\
  D \mbox{ (definitions) } &::=& \langle d_1 \cup \dots \cup d_n \rangle\\
  P \mbox{ (programs)} &::=& \langle D, s \rangle\\
 \end{eqnarray*}
\caption{Syntax of $\mathcal{L}$.} \label{fig:syntax}
\end{figure*}
%  and for simplification we only use pointers as values.

% and each variable is a pointer.

The syntax of the language \(\mathcal{L}\) is defined by the BNF in
Figure~\ref{fig:syntax}.  The \(\VAR\) is a countably infinite set of
\emph{variables}. The statement \(\SKIP\) is a do-nothing statement.
The statement \(s_1;s_2\) is a sequential execution of \(s_1\) and
\(s_2\). The statement \(*x \leftarrow y\) changes the content of cell
which is pointed to by \(x\) to the value \(y\). The statement
\(\Free(x)\) deallocates a memory cell pointed to by the pointer
\(x\). The statement \(\LET x = e\ \IN s\) evaluates the expression
\(e\), binds \(x\) to the result, and executes \(s\).  The expression
\(\Malloc()\) allocates a new memory cell.  We assume that the size of
every memory cell is identical.  The expression \(\NULL\) evaluates to
the null pointer.  The expression $*y$ evaluates to the content of the
memory cell pointed to by \(y\).  The statement \(\IFNULL(*x) \; \THEN s_1
\; \ELSE s_2\) executes \(s_1\) if \(*x\) is \(\NULL\) and executes \(s_2\)
otherwise. The statement \(f(\vec{x})\) executes the procedure \(f\)
with arguments \(\vec{x}\).  For simplicity, a value is either a pointer
or $\NULL$.

Our type system, as we described in Section~\ref{sec:introduction},
needs to keep track of whether $*x$ is null or not for a given $x$.  To
this end, we introduce \emph{constantness annotation} \(\scon\Sirx s\)
to our language.  By writing this annotation, a programmer declares that
the value of $*x$ does not change during the execution of $s$.  The
statement \(\Endconst\) is a marker for the end of a constantness
annotation, which is used only in a runtime state.

A \emph{procedure definition} $f \mapsto (x_1,\dots,x_n)s$ defines a
procedure $f$ that takes $x_1,\dots,x_n$ as arguments and executes $s$.
We use metavariable \(D\) for a set of procedure definitions.  A program
is a pair \(\langle D, s \rangle \), where \(D\) is the set of the
procedure definitions that may be used in the execution of the main
statement $s$.

\subsection{Operational semantics}
\label{sec:languageSemantics}

\begin{myDef}
 A \emph{fact} $c$ is defined by the following BNF:
 \[
 c ::= \scon\Sirx \mid \snull \mid \snnull.
 \]
 We use $C$ for a multiset of facts.
\end{myDef}
A fact is used in the semantics to keep track of the information about
constantness and nullness.  The fact \(\scon\Sirx\) represents that $*x$
is declared to be currently constant by a constantness annotation;
\(\snull\) and \(\snnull\) represents that $*x$ is equal to $\NULL$ and
not equal to $\NULL$ respectively.

For definition of the operational semantics, we designate a countable
infinite set \(\mathcal{H}\) of \(\emph{heap addresses}\) ranged over by
\(l\).  A configuration is of the form \(\langle H, R, s, n, C
\rangle\).  Each elements in the configuration is as follows:
\begin{itemize}
\item \(H\), a \emph{heap}, is a finite mapping from \(\mathcal{H}\) to
  \(\mathcal{H} \cup \set{\NULL}\) representing the state of memory,
\item \(R\), an \emph{environment}, is a finite mapping from \(\VAR\)
  to \(\mathcal{H} \cup \set{\NULL}\) representing the value bound to
      each variable,
\item \(s\) is the statement that is being executed,
\item \(n\) is the number of the memory cells available for allocation,
      and
\item \(C\) is a multiset of \emph{facts} that holds at the current
      configuration.
\end{itemize}

The operational semantics of the language \(\mathcal{L}\) is given by a
labeled transition relation \(\langle H, R, s, n, C \rangle
\xlongrightarrow{\rho}_D \langle H', R', s', n', C' \rangle\) where
\(\rho\) is a label defined by the BNF $\rho ::= \Malloc \mid \Free \mid
\snull \mid \snnull \mid \tau$.  The label \(\Malloc\) expresses an
allocation of a new memory cell.  The label \(\Free\) expresses a
deallocation of a memory cell.  The label \(\snull\) expresses making an
assumption that \(*x\) is a null pointer; the label $\snnull$ expresses
not.  The label \(\tau\) expresses an action other than the above; we
often omit \(\tau\) in the relation \(\xlongrightarrow{\tau}_D\).  The
metavariable \(\sigma\) is used for a finite sequence of actions
\(\rho_1\dots\rho_n\). The \(\xlongrightarrow{\rho_1\dots\rho_n}_D\) is
short for
\(\xlongrightarrow{\rho_1}_D\xlongrightarrow{\rho_2}_D\dots\xlongrightarrow{\rho_n}_D\).
We write \(\xLongrightarrow{\rho}_D\) for
\(\xlongrightarrow{}_D^*\xlongrightarrow{\rho}_D\xlongrightarrow{}_D^*\).
We write \(\xLongrightarrow{\rho_1\dots\rho_n}_D\) for
\(\xLongrightarrow{\rho_1}_D\dots\xLongrightarrow{\rho_n}_D\).

\begin{figure*}
\footnotesize
  \begin{minipage}{\textwidth}

\infax[Sem-Skip]
{\langle H, R, \SKIP;s, n, C \rangle
\longrightarrow_{D}
\langle H, R, s, n, C \rangle}

\rulesp

\infrule[Sem-Seq]
{\langle H, R, s_1, n, C  \rangle \xlongrightarrow{\rho}_{D} \langle H', R', s_1', n', C' \rangle}
{\langle H, R, s_1;s_2, n, C \rangle \xlongrightarrow{\rho}_{D} \langle H', R', s_1';s_2, n', C' \rangle}
 
\rulesp

\infrule[Sem-LetNull]
{x' \notin \DOM(R)}
{\langle H\coma R\coma  \LET x = \NULL \ \IN s , n, C \rangle
  \longrightarrow_{D}
  \langle H\coma R\Lfc x' \mapsto \NULL \Rfc \coma   \Lb x'/x \Rb s , n, C  \rangle }

\rulesp

\infrule[Sem-LetEq]
{x' \notin \DOM(R)}
{\langle H\coma R\coma \LET x = y \; \IN s , n, C  \rangle
  \longrightarrow_{D}
  \langle H\coma R\Lfc x' \mapsto R(y) \Rfc \coma   \Lb x'/x \Rb s , n, C\rangle }

\rulesp

\infax[Sem-ConstSeq]
{\langle H, R, \scon\Sirx s, n, C \rangle
\rightarrow_{D}
\langle H, R, s;\Endconst , n, C \cup \{\scon\Sirx\} \rangle}

\rulesp

 \infrule[Sem-ConstSkip]
   {C' = \FILTER(C, *x)}
   {\langle H, R,
     \Endconst, n, C \rangle \rightarrow_{D} \langle
     H, R, \SKIP , n, C' \rangle}

\rulesp

\infrule[Sem-Assign]
{\forall z. R(x) = R(z) \implies \scon(z) \notin C}
{ \langle H \set{R(x) \mapsto v}, R, *x \leftarrow y , n, C\rangle \xlongrightarrow{}_{D}
  \langle H \Lfc R(x) \mapsto R(y) \Rfc , R, \SKIP , n, C \rangle }

\rulesp

\infrule[Sem-IfNullT]
{H(R(x)) = \NULL \andalso \scon\Sirx \notin C}
{\langle H \coma R \coma \IFNULL\Sirx \ \THEN   s_{1}\ \ELSE\  s_{2} \coma  n, C \rangle
  \xlongrightarrow{\snull}_D
  \langle H\coma R\coma s_{1} \coma n, C \rangle}

\rulesp

\infrule[Sem-IfNullF]
{H(R(x)) \neq \NULL \andalso \scon\Sirx \notin C}
{\langle H \coma R \coma \IFNULL\Sirx\ \THEN  s_{1}\ \ELSE  s_{2} \coma  n, C\rangle
  \xlongrightarrow{\snnull}_D
  \langle H\coma R\coma s_{2} \coma  n, C\rangle}


\rulesp

\infrule[Sem-IfConstNullT]
{H(R(x)) = \NULL \andalso \scon\Sirx \in C }
{\langle H \coma R \coma \IFNULL\Sirx \ \THEN   s_{1}\ \ELSE\  s_{2} \coma  n, C \rangle
  \xlongrightarrow{\snull}_D
  \langle H\coma R\coma s_{1} \coma n, C\cup \{\snull\} \rangle}

\rulesp

\infrule[Sem-IfConstNullF]
{H(R(x)) \neq \NULL \andalso \scon\Sirx \in C }
{\langle H \coma R \coma \IFNULL\Sirx\ \THEN  s_{1}\ \ELSE  s_{2} \coma  n, C\rangle
  \xlongrightarrow{\snnull}_D
  \langle H\coma R\coma s_{2} \coma  n, C\cup \{\snnull\} \rangle}

\rulesp

\infrule[Sem-LetDeref]
{x' \notin \DOM(R) \andalso R(y) \in \DOM(H)}
{\langle H\coma R\coma  \LET x = *y \; \IN s , n, C \rangle
  \longrightarrow_{D}
  \langle H\coma R\Lfc x' \mapsto H(R(y)) \Rfc \coma   \Lb x'/x \Rb s , n, C\rangle }

\rulesp

\infrule[Sem-Free]
{R(x) \neq \NULL \mbox{ and } R(x) \in \DOM(H)}
{\langle H\set{R(x) \mapsto v}\coma R\coma \Free(x) , n, C \rangle \xlongrightarrow{\Free}_{D}
  \langle H\backslash R(x) \coma R \coma \SKIP , n+1, C \rangle}

\rulesp

\infrule[Sem-Malloc]
{l \notin \DOM(H) \andalso n > 0}
{\langle H\coma R\coma  \LET x = \Malloc() \; \IN s , n, C\rangle
  \xlongrightarrow{\Malloc}_{D}
  \langle H \Lfc l \mapsto v\Rfc \coma R\Lfc x' \mapsto l \Rfc \coma   \Lb x'/x \Rb s , n-1, C  \rangle }

\rulesp
\begin{minipage}{0.5\textwidth}
\infrule[Sem-Call]
{D(f) = (\vec{y})s}
{ \langle H\coma R\coma  f(\vec{x}) , n, C\rangle
  \longrightarrow_{D}
  \langle H\coma R\coma  \Lb \vec{x}/\vec{y} \Rb s , n, C \rangle}
\end{minipage}
\begin{minipage}{0.5\textwidth}
\infrule[Sem-FreeExn]
{R(x) = \NULL \mbox{ or } R(x) \notin \DOM(H)}
{\langle H\coma R\coma \Free(x) , n, C\rangle \xlongrightarrow{\Free}_{D} \MEMEX}
\end{minipage}
\rulesp

\begin{minipage}{0.5\textwidth}
\infrule[Sem-AssignExn]
{R(x) = \NULL \mbox{ or } R(x) \notin \DOM(H)}
{\langle H\coma R\coma  *x \leftarrow y , n, C\rangle
  \longrightarrow_{D} \MEMEX }
\end{minipage}
\begin{minipage}{0.5\textwidth}
\infrule[Sem-DerefExn]
{R(y) = \NULL \mbox{ or } R(y) \notin \DOM(H)}
{\langle H\coma R\coma  \LET x = *y \; \IN s, n, C\rangle
    \longrightarrow_{D} \MEMEX}
\end{minipage}
\rulesp

\infrule[Sem-AssignConstExn]
{ \exists z. \scon(*z) \in C \mbox{ and } R(x) = R(z)}
{ \langle H \set{R(x) \mapsto v}, R, *x \leftarrow y , n, C\rangle \xlongrightarrow{}_{D}
  \CONSTEX}

%% \infrule[Sem-ConstExn]
%%         { H(R(x)) \neq H'(R'(x))
%%           \Rtab
%%           \langle H, R, s, n, C \rangle \xlongrightarrow{\rho}_{D} \langle H', R', s', n', C \rangle}
%% {\langle H, R, \scon\Sirx s, n, C \rangle \xlongrightarrow{\rho}_{D} \CONSTEX }

\infax[Sem-OutOfMem]
{ \langle H\coma R\coma \LET x = \Malloc() \ \IN s ,  0, C  \rangle    \xlongrightarrow{\Malloc}_{D}
  \OVERFLOW}

\end{minipage}

\caption{Operational semantics of \(\mathcal{L}\).}
\label{fig:transitionRules}
\end{figure*}

The relation $\xlongrightarrow{\rho}_D$ is the least relation that
satisfies the rules in Figure~\ref{fig:transitionRules}.  In these
rules, $E$ is a \emph{context} defined by $E ::= E;s \mid []$.

The operational semantics is, except for the multiset of facts $C$,
almost the same as the previous framework~\cite{tanPPL2014}.  Important
invariants of a configuration is that (1) if $C$ contains $\scon(*x)$,
then the value of the expression $*x$ is not changed in the next step
and (2) if $C$ contains $\snull$ (resp., $\snnull$), then the value of
$*x$ is equal to $\NULL$ (resp., not equal to null).

In order to manipulate the $C$ part of a configuration, we use a
function $\FILTER(C, *x)$ in the rules for the operational semantics
defined as follows:
\[
\begin{array}{l}
 \FILTER(C, *x) =\\
 \quad \LET\ C' = C - \scon\Sirx\ \IN\\
 \quad\quad \IF\ \scon\Sirx \in C'\ \THEN\ C'\\
 \quad\quad \ELSE\ C'\backslash\{\snull,\snnull\}.
\end{array}
\]
The function $\FILTER(C, *x)$ sets $C'$ to the multiset obtained by
removing one $\scon(*x)$ from $C$.  (Recall that $C$ is a
\emph{multiset} of facts.)  If $C'$ does not contain $\scon(*x)$, then
it deletes all the facts related to $*x$.  Otherwise, it returns $C'$.

We explain the rules related to path sensitivity and the constantness
annotations; for the other rules, see~\ref{tanPPL2014}.
\begin{itemize}
\item \textsc{Sem-ConstSeq}: The end of the $\mathbf{const}$ block is
      marked at the end of the current continuation.  The fact
      $\scon(*x)$ is also added to $C$ to record that $*x$ is declared
      to be constant until $\Endconst$ is encountered.
\item \textsc{Sem-Assign} and \textsc{Sem-ConstAssignExn}: These rules
      handle assignment statements.  If constantness of $*z$ is declared
      (i.e., $\scon(*z) \in C$), then a write operation to $*z$ leads
      to an exception $\CONSTEX$.  This exception is raised if a
      constantness annotation provided by a programmer is wrong.  The
      type system introduced later does not guarantee unreachability to
      this exceptional state.
\item \textsc{Sem-IfNullT}, \textsc{Sem-IfNullF},
      \textsc{Sem-IfConstNullT}, and \textsc{Sem-IfConstNullF}: These
      rules handle an $\mathbf{if}$-statements depending on whether $*x$
      is null or not.  The semantics records which branch is taken to
      the multiset of facts $C$ only if $C$ contains $\scon\Sirx$.
      Otherwise, even if the semantics recorded $\snull$ or
      $\snnull$ to $C$, it may not be reliable in the sense of the
      invariant (2) above since the value of $*x$ may be changed by an
      assignment via an alias of $x$.
\item \textsc{Sem-ConstSkip}: By using the function $\FILTER$ above,
      this rule removes one $\mathbf{endconst}(*x)$ from $C$ and
      conducts ``garbage collection'' of the facts related to $*x$ if
      necessary.
\item \textsc{Sem-Malloc} and \textsc{Sem-OutOfMem}: $\Malloc$ allocates
      a new memory cell by extending the heap $H$ and decreasing the
      value of $n$ by $1$.  If there is no available cell (i.e., $n =
      0$), then $\Malloc$ raises $\OVERFLOW$ exception.
\item \textsc{Sem-Free}: $\Free(x)$ removes the entry for $R(x)$ from
      the heap $H$ and increase the value of $n$ by $1$.
\item \textsc{Sem-AssignExn}, \textsc{Sem-FreeExn} and
      \textsc{Sem-DerefExn}: If a program tries to access to a
      deallocated memory cell, then the exception $\MEMEX$ is raised.
\end{itemize}

\begin{myDef}
\label{df:ml} A program $\langle D, s \rangle$ is said to \emph{execute
within $n$ memory cells} if $\langle \emptyset, \emptyset, s, n,
\emptyset\rangle \not\xlongrightarrow{}_D^* \MEMEX$.
\end{myDef}


% In order to deal with a path-sensitive program to guarantee
% \emph{total} memory-leak freedom, we redefined the several definitions
% as follows.  defined as follows:
% \begin{myDef}[total memory-leak freedom]
% \label{df:ml}
% A program \(\langle D, s \rangle\) is \emph{totally memory-leak free}
% if there is a natural number \(n\) such that it does not require more
% than \(n\) cells.
% \end{myDef}
% \begin{myDef}[Memory leak]
% \label{df:ml}
% A configuration \(\langle H, R, s, n, C \rangle\) \emph{goes overflow} if
% there is \(\sigma\) such that \(\langle H, R, s, n, C \rangle
% \xLongrightarrow{\sigma} \OVERFLOW\).  A program \(\langle D, s
% \rangle\) \emph{consumes at least \(n\) cells} if \(\langle \emptyset,
% \emptyset, s, n, \emptyset \rangle\) goes overflow.  
% \end{myDef}


