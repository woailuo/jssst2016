% 以下の3行は変更しないこと.
\documentclass[T]{compsoft}
\taikai{2016}
\pagestyle {empty}

\usepackage [dvipdfmx] {graphicx}
\usepackage{extarrows}
%% \usepackage{bcprules,amsmath,amsthm,amssymb,amsfonts,extarrows,geometry,amsopn,enumerate,xcolor,url}
%% \usepackage[sort]{cite}
%% \usepackage{graphicx}
%% \usepackage{boxedminipage}
%% \usepackage{url}
%% \usepackage{multirow}

% ユーザが定義したマクロなどはここに置く.ただし学会誌のスタイルの
% 再定義は原則として避けること.

% set some new commands %

\newcommand\tB{\;|\;}
\newcommand\LET{\mathbf{let}\;}
\newcommand\FREE{\mathbf{free}(x)\;}
\newcommand\IN{\mathbf{in}\;}
\newcommand\SKIP{\mathbf{skip}}
\newcommand\Rtab{\; \; \; \;}
\newcommand\NULL{\mathbf{null}}
\newcommand\IFNULL{\mathbf{ifnull}\;}
\newcommand\THEN{\mathbf{then}\;}
\newcommand\ELSE{\mathbf{else}\;}
\newcommand\Lcc{\left(}
\newcommand\Rcc{\right)}
\newcommand\Lfc{\left\{}
\newcommand\Rfc{\right\}}
\newcommand\Lb{\left[}
\newcommand\Rb{\right]}
\newcommand\coma{,\;}
\newcommand\MALLOC{\mathbf{malloc()}\;}
\newcommand\Malloc{\mathbf{malloc}}
\newcommand\Free{\mathbf{free}}
\newcommand\Cirx{(x)}
\newcommand\Sirx{(*x)}
\newcommand\dtb{\;\;\ \;\;\ \;\;\ \;\;\  }
\newcommand\set[1]{\{#1\}}
\newcommand\VAR{\mathbf{Var}}
\newcommand\OK{\mathit{OK}}
\newcommand\COL{\!:\!}
\newcommand\TSEQ{;\!}
\newcommand\TSKIP{\mathbf{0}}
\newcommand\OVERFLOW{\mathbf{OutOfMemory}}
\newcommand\DOM{\mathbf{Dom}}
\newcommand\FUNTYPE{\varphi}
\newcommand\bs{\backslash}
\newcommand\MEMEX{\mathbf{MemEx}}
\newcommand\CONSTEX{\mathbf{ConstEx}}

\newcommand\xnull{*x = \NULL}
\newcommand\xnnull{*x \ne \NULL}
%% \newcommand\assx{\mathbf{assume(x = null)}}
%% \newcommand\assxn{\mathbf{assume(x != null)}}
\newcommand\const{\mathbf{const}(\sigma, n)}
\newcommand\snull{{\bf null}\Sirx}
\newcommand\snnull{\neg{\bf null}\Sirx}

\newcommand\sassx{\mathbf{assume}(\xnull)}
\newcommand\sassxn{\mathbf{assume}(\xnnull)}
\newcommand\sconst{\mathbf{consistency}(\sigma)}
\newcommand\sconstr{\mathbf{consistency}(\rho)}
\newcommand\scon{\mathbf{const}}
\newcommand\Startconst{\mathbf{startconst}\Sirx}
\newcommand\Endconst{\mathbf{endconst}\Sirx}

\newtheorem{theorem}{Theorem}[section]
\newtheorem{lemma}[theorem]{Lemma}
\newtheorem{proposition}[theorem]{Proposition}
\newtheorem{corollary}[theorem]{Corollary}
\newtheorem{myDef}{Definition}
\newtheorem{remark}{Remark}[section]

%% \renewcommand\rn[1]{\textsc{{#1}}}
%% \theoremstyle{definition}
\newtheorem{exmp}{Example}[section]

%% \newenvironment{nospaceflalign*}
%%  {\setlength{\abovedisplayskip}{0pt}\setlength{\belowdisplayskip}{0pt}%
%%   \csname flalign*\endcsname}
%%  {\csname endflalign*\endcsname\ignorespacesafterend}


\begin{document}

% 論文のタイトル
\title{An Extended Behavioral Type System \\ for Memory-Leak Freedom}

% 著者
% 和文論文の場合,姓と名の間には半角スペースを入れ,
% 複数の著者の間は全角スペースで区切る
%
\author{Qi Tan Kohei Suenaga Atsushi Igarashi
%
% ここにタイトル英訳 (英文の場合は和訳) を書く.
%
\ejtitle{Concurrent Operations on Splay Trees.}
%
% ここに著者英文表記 (英文の場合は和文表記) および
% 所属 (和文および英文) を書く.
% 複数著者の所属はまとめてよい.
%
\shozoku{for}{example}%
{Dept.\ of Information and Computer Science, Waseda University}}

% 和文アブストラクト
\Jabstract{%
%% SleatorとTarjanによる自己調整二分木(splay tree)に対して並
%% 列操作を可能にする操作アルゴリズムを提案する.提案するアル
%% ゴリズムは,同一の木に対する複数の更新・挿入・削除操作のパイプライン的並
%% 列実行を許し,かつ操作系列のスループット(単位時間内に処理可能な
%% 操作の個数)とレスポンス(個々の操作の償却計算量
%% (amortized complexity))を両立させることを目的としている.
%% スループットの最適性と挿入操作の対数的レスポンスについては
%% 理論的結果を示す.削除操作は,木の形状に関する良い性質を保つ
%% にもかかわらず,Sleatorらの枠組みでは最適性が証明できない.
%% このことについても論じる.
}

% 英文アブストラクト(本サンプルの原論文にはなし)
\Eabstract{
  %% \section{Abstract}
\label{sec:abstraction}
In the previous work, we proposed a behavioral type system for a
programming language with dynamic memory allocation and deallocation.
The behavioral type system, which uses sequential processes as types
where each action is related to an allocation and a deallocation, can
estimate an upper bound of memory consumption of a program.  However,
the previous type system did not deal with path-sensitivity, which
results in an imprecise abstraction even for a simple program.  In
order to address this problem, we propose an extension of the previous
type system with dependent types.  The dependent type carries more
information for a program such that it can handle path-sensitivity and
estimate an upper bound of memory cell consumption more precisely.  We
prove the soundness of the extended type system and propose a type
inference algorithm of this type system.  We also implemented the
algorithm.  Our experiment shows that the extended type system is very
useful to deal with a practical path-sensitive program by checking
whether the if-guard-part is a null pointer or not.

}
%
\maketitle \thispagestyle {empty}

\section{Introduction}
\label{sec:introduction}
Manual memory mangagement primitives (e.g. \(\texttt{malloc}\) and
\(\texttt{free}\) in C language) are a very flexible way to manage
computer memory cells.  We can write a program which dynamically
allocates a memory cell during running and deallocates a memory cell
when it is no longer used. However, manual memory management
primitives often cause hard-to-find problems, for example, double
frees (\texttt{free} a deallocated memory cell ), memory leaks (forget
to deallocate memory cells) and illegal accesses to a dangling
pointer. Therefore, many static verification methods have been
proposed to guarantee safe memory deallocation. They prove
\emph{partial} memory-leak freedom: if a program terminates, all the
memory cells are safe deallocated. As we know that nonterminating
programs are very common in real-world programmings such as Web
servers and operating systems. To guarantee \emph{total} memory-leak
freedom, if a program does not consume unbounded number of memory
cells during execution, is a very crucial issue.

\begin{exmp}\label{ex:ex1}
\begin{figure}[h]
\caption{Memory leaks in nonterminating programs.}
\label{ex:np}
\end{figure}
Figure~\ref{ex:np} describes partial and total memory-leak freedom.
Both \(h\) and \(h'\) are partially memory-leak free because they do
not terminate.  The function \(h\) is totally memory-leak free since
it consumes at most two cells\footnote{We assume that every memory
  cell allocated by \(\Malloc\) is fixed size to simplify our type
  system introduced in Section~\ref{sec:typesystem}.  Extension with
  variable-length cells is one of our future work.}.  However, the
function \(h'\), when it is invoked, consumes unbounded number of
memory cells; hence \(h'\) is not totally memory-leak free.
\end{exmp}

In order to prove \emph{total} memory deallocation, we proposed a
behavioral type system in previous study[]. It can abstract the
behavior of a program by using sequential process whose actions
represent manual memory management primitives, and our behavior type
only consider the the number and order of manual memory management
primitives and recursively calls. For example, the abstract behavioral
type of function \(h\) in Figure~\ref{ex:np} is
\(\mu\alpha.\Malloc;\Malloc;\Free;\Free;\alpha\), which represents
function \(h\) allocates two memory cells, deallocates them, and then
recursively call itself again; the behavioral type of function \(h'\)
is abstracted as \(\mu\alpha.\Malloc;\Malloc;\alpha;\Free;\Free\),
which represents \(h'\) allocates two memory cells, call itself again,
and then deallocates those two cells. That way we can easily estimate
the upper bound of memory cells a program consumed.

Although our previours behavioral type system can abstract the
behavior of a program and estimate the upper bound of memory
consumption, verification on abstracted behavioral types are failed in
some cases. For example, the extracted behavioral type of function
\(foo\) in Figure~\ref{ex:np2} is \( \mu\alpha.\Malloc;\Malloc;\Malloc
+ \TSKIP;\Free + \TSKIP;\Free;\Free;\alpha \), which expresses that
function \(foo\) allocates two memory cells, a choice command between
allocating one memory cell and skipping, a choice command between
deallocating one cell and skipping, deallocates two cells, and then
call itself again. Due to the choice behavioral type, the above type
may be seen as \(
\mu\alpha.\Malloc;\Malloc;\Malloc;\TSKIP;\Free;\Free;\alpha \), which
expresses function \(foo\) consumes three memory cells but deallocates
two memory cells, and then iterates this behavior again. This behavior
means function \(foo\) consumes unbounded number of memory cells,
although the original program is \emph{total} memory-leak freedom.

\begin{exmp}\label{ex:ex2}
\begin{figure}[h]
\caption{a nonterminating program with conditionals}
\label{ex:np2}
\end{figure}
Figure~\ref{ex:np2} describes that functioin \(foo\) is a total
memory-leak freedom program, because it consumes at most three memory
cells during execution. This function has two conditionals: if \(*y\)
is not a null pointer, it will allocates one cell at first conditional
and deallocates that cell at second conditional, otherwise skips.
\end{exmp}

Our current idea is to extend previous behavioral type system with
dependent types[]. The dependent types takes more precise information
than traditional types, for example, the type
\((*x)(\Malloc, \TSKIP)\) is a dependent type, because it dependents
on the value \((*x)\). See the function \(foo\) again, the current
behavioral type of it is \(
\mu\alpha.\Malloc;\Malloc;(*x)(\Malloc, \TSKIP);(*x)(\Free, \TSKIP);\\ \Free;\Free;\alpha
\). Therefore, the part \((*x)(\Malloc, \TSKIP);\\(*x)(\Free, \TSKIP)\)
can be seen as \(\Malloc;\Free\) or \(\TSKIP;\TSKIP\) if \((*x)\) does
not change between these two choices, which we can definitely judge it
is a total memory-leak free program.

The reminder of this paper is structured as
follows. Section~\ref{sec:language} describes an imperative language
with allocation and deallocation primitives and its operational
semantics. Section~\ref{sec:typesystem} introduces the extended
behavioral type system with dependent
types. Section~\ref{sec:reconstruction} describes the type
reconstruction procedure; Section~\ref{sec:experiments} shows some
experiments and give a discussion; Section~\ref{sec:relatedwork}
describes the related works; Section~\ref{sec:conclusion} concludes
this paper.

\section{Language \(\mathcal{L}\)}\label{sec:language}

This section defines an imperative language \(\mathcal{L}\) with memory
allocation and deallocation primitives.  

In the rest of this paper, we write \(\vec{x}\) for a finite sequence
\(x_1,\dots,x_n\); we assume that each element is distinct.  We write
\([\vec{x'}/\vec{x}]s\) for the term obtained by replacing each free
occurrence of \(\vec{x}\) in $s$ with variables \(\vec{x'}\).  We write
\(\DOM(f)\) for the partial function $f$.  We write \(f \set{x \mapsto
v}\) and \(f \bs x\) as follows.
\[
\begin{array}{rcl}
f \set{x \mapsto v} (w) &=&
\left\{
\begin{array}{ll}
v & \mbox{if \(x = w\)}\\
f(w) & \mbox{otherwise.}
\end{array}
\right.\\
(f \bs x)(w) &=&
\left\{
\begin{array}{ll}
\mbox{undefined} & \mbox{if \(x = w\)}\\
f(w) & \mbox{otherwise.}
\end{array}
\right.
\end{array}
\]


\subsection{Syntax}

\begin{figure*}[t]
 \begin{eqnarray*}
  x,y,z,\dots \mbox{ (variables)} &\in& \VAR\\
  s \mbox{ (statements)} & ::= &  \SKIP \mid s_{1};s_{2} \mid *x \leftarrow y \mid \Free(x) \\
  & \mid & \LET x = \MALLOC \IN s \mid \LET x = \NULL\ \IN s  \\
  & \mid & \LET x = y \; \IN s \mid   \LET x = *y \; \IN s \\
  & \mid & \IFNULL(*x) \; \THEN s_{1}\; \ELSE s_{2} \mid f(\vec{x})\\
  & \mid & \scon\Sirx s \mid \Endconst  \\
  d \mbox{ (proc. defs.)} & ::= & \set{f \mapsto (x_1,\dots,x_n)s}\\
  D \mbox{ (definitions) } &::=& \langle d_1 \cup \dots \cup d_n \rangle\\
  P \mbox{ (programs)} &::=& \langle D, s \rangle\\
 \end{eqnarray*}
\caption{Syntax of $\mathcal{L}$.} \label{fig:syntax}
\end{figure*}
%  and for simplification we only use pointers as values.

% and each variable is a pointer.

The syntax of the language \(\mathcal{L}\) is defined by the BNF in
Figure~\ref{fig:syntax}.  The \(\VAR\) is a countably infinite set of
\emph{variables}. The statement \(\SKIP\) is a do-nothing statement.
The statement \(s_1;s_2\) is a sequential execution of \(s_1\) and
\(s_2\). The statement \(*x \leftarrow y\) changes the content of cell
which is pointed to by \(x\) to the value \(y\). The statement
\(\Free(x)\) deallocates a memory cell pointed to by the pointer
\(x\). The statement \(\LET x = e\ \IN s\) evaluates the expression
\(e\), binds \(x\) to the result, and executes \(s\).  The expression
\(\Malloc()\) allocates a new memory cell.  We assume that the size of
every memory cell is identical.  The expression \(\NULL\) evaluates to
the null pointer.  The expression $*y$ evaluates to the content of the
memory cell pointed to by \(y\).  The statement \(\IFNULL(*x) \THEN s_1
\ELSE s_2\) executes \(s_1\) if \(*x\) is \(\NULL\) and executes \(s_2\)
otherwise. The statement \(f(\vec{x})\) executes the procedure \(f\)
with arguments \(\vec{x}\).  For simplicity, a value is either a pointer
or $\NULL$.

Our type system, as we described in Section~\ref{sec:introduction},
needs to keep track of whether $*x$ is null or not for a given $x$.  To
this end, we introduce \emph{constantness annotation} \(\scon\Sirx s\)
to our language.  By writing this annotation, a programmer declares that
the value of $*x$ does not change during the execution of $s$.  The
statement \(\Endconst\) is a marker for the end of a constantness
annotation, which is used only in a runtime state.

A \emph{procedure definition} $f \mapsto (x_1,\dots,x_n)s$ defines a
procedure $f$ that takes $x_1,\dots,x_n$ as arguments and executes $s$.
We use metavariable \(D\) for a set of procedure definitions.  A program
is a pair \(\langle D, s \rangle \), where \(D\) is the set of the
procedure definitions that may be used in the execution of the main
statement $s$.

\subsection{Operational semantics}
\label{sec:languageSemantics}

\begin{myDef}
 A \emph{fact} $c$ is defined by the following BNF:
 \[
 c ::= \scon\Sirx \mid \snull \mid \snnull.
 \]
 We use $C$ for a multiset of facts.
\end{myDef}
A fact is used in the semantics to keep track of the information about
constantness and nullness.  The fact \(\scon\Sirx\) represents that $*x$
is declared to be currently constant by a constantness annotation;
\(\snull\) and \(\snnull\) represents that $*x$ is equal to $\NULL$ and
not equal to $\NULL$ respectively.

For definition of the operational semantics, we designate a countable
infinite set \(\mathcal{H}\) of \(\emph{heap addresses}\) ranged over by
\(l\).  A configuration is of the form \(\langle H, R, s, n, C
\rangle\).  Each elements in the configuration is as follows:
\begin{itemize}
\item \(H\), a \emph{heap}, is a finite mapping from \(\mathcal{H}\) to
  \(\mathcal{H} \cup \set{\NULL}\) representing the state of memory,
\item \(R\), an \emph{environment}, is a finite mapping from \(\VAR\)
  to \(\mathcal{H} \cup \set{\NULL}\) representing the value bound to
      each variable,
\item \(s\) is the statement that is being executed,
\item \(n\) is the number of the memory cells available for allocation,
      and
\item \(C\) is a multiset of \emph{facts} that holds at the current
      configuration.
\end{itemize}

The operational semantics of the language \(\mathcal{L}\) is given by a
labeled transition relation \(\langle H, R, s, n, C \rangle
\xlongrightarrow{\rho}_D \langle H', R', s', n', C' \rangle\) where
\(\rho\) is a label defined by the BNF $\rho ::= \Malloc \mid \Free \mid
\snull \mid \snnull \mid \tau$.  The label \(\Malloc\) expresses an
allocation of a new memory cell.  The label \(\Free\) expresses a
deallocation of a memory cell.  The label \(\snull\) expresses making an
assumption that \(*x\) is a null pointer; the label $\snnull$ expresses
not.  The label \(\tau\) expresses an action other than the above.; we
often omit \(\tau\) in the relation \(\xlongrightarrow{\tau}_D\).  The
metavariable \(\sigma\) is used for a finite sequence of actions
\(\rho_1\dots\rho_n\). The \(\xlongrightarrow{\rho_1\dots\rho_n}_D\) is
short for
\(\xlongrightarrow{\rho_1}_D\xlongrightarrow{\rho_2}_D\dots\xlongrightarrow{\rho_n}_D\).
We write \(\xLongrightarrow{\rho}_D\) for
\(\xlongrightarrow{}_D^*\xlongrightarrow{\rho}_D\xlongrightarrow{}_D^*\).
We write \(\xLongrightarrow{\rho_1\dots\rho_n}_D\) for
\(\xLongrightarrow{\rho_1}_D\dots\xLongrightarrow{\rho_n}_D\).

\begin{figure*}
\footnotesize
  \begin{minipage}{\textwidth}

\infax[Sem-Skip]
{\langle H, R, \SKIP;s, n, C \rangle
\longrightarrow_{D}
\langle H, R, s, n, C \rangle}

\rulesp

\infrule[Sem-Seq]
{\langle H, R, s_1, n, C  \rangle \xlongrightarrow{\rho}_{D} \langle H', R', s_1', n', C' \rangle}
{\langle H, R, s_1;s_2, n, C \rangle \xlongrightarrow{\rho}_{D} \langle H', R', s_1';s_2, n', C' \rangle}
 
\rulesp

\infrule[Sem-LetNull]
{x' \notin \DOM(R)}
{\langle H\coma R\coma  \LET x = \NULL \ \IN s , n, C \rangle
  \longrightarrow_{D}
  \langle H\coma R\Lfc x' \mapsto \NULL \Rfc \coma   \Lb x'/x \Rb s , n, C  \rangle }

\rulesp

\infrule[Sem-LetEq]
{x' \notin \DOM(R)}
{\langle H\coma R\coma \LET x = y \; \IN s , n, C  \rangle
  \longrightarrow_{D}
  \langle H\coma R\Lfc x' \mapsto R(y) \Rfc \coma   \Lb x'/x \Rb s , n, C\rangle }

\rulesp

\infax[Sem-ConstSeq]
{\langle H, R, \scon\Sirx s, n, C \rangle
\rightarrow_{D}
\langle H, R, s;\Endconst , n, C \cup \{\scon\Sirx\} \rangle}

\rulesp

 \infrule[Sem-ConstSkip]
   {C' = \FILTER(C, *x)}
   {\langle H, R,
     \Endconst, n, C \rangle \rightarrow_{D} \langle
     H, R, \SKIP , n, C' \rangle}

\rulesp

\infrule[Sem-Assign]
{\forall z. R(x) = R(z) \implies \scon(z) \notin C}
{ \langle H \set{R(x) \mapsto v}, R, *x \leftarrow y , n, C\rangle \xlongrightarrow{}_{D}
  \langle H \Lfc R(x) \mapsto R(y) \Rfc , R, \SKIP , n, C \rangle }

\rulesp

\infrule[Sem-IfNullT]
{H(R(x)) = \NULL \andalso \scon\Sirx \notin C}
{\langle H \coma R \coma \IFNULL\Sirx \ \THEN   s_{1}\ \ELSE\  s_{2} \coma  n, C \rangle
  \xlongrightarrow{\snull}_D
  \langle H\coma R\coma s_{1} \coma n, C \rangle}

\rulesp

\infrule[Sem-IfNullF]
{H(R(x)) \neq \NULL \andalso \scon\Sirx \notin C}
{\langle H \coma R \coma \IFNULL\Sirx\ \THEN  s_{1}\ \ELSE  s_{2} \coma  n, C\rangle
  \xlongrightarrow{\snnull}_D
  \langle H\coma R\coma s_{2} \coma  n, C\rangle}


\rulesp

\infrule[Sem-IfConstNullT]
{H(R(x)) = \NULL \andalso \scon\Sirx \in C }
{\langle H \coma R \coma \IFNULL\Sirx \ \THEN   s_{1}\ \ELSE\  s_{2} \coma  n, C \rangle
  \xlongrightarrow{\snull}_D
  \langle H\coma R\coma s_{1} \coma n, C\cup \{\snull\} \rangle}

\rulesp

\infrule[Sem-IfConstNullF]
{H(R(x)) \neq \NULL \andalso \scon\Sirx \in C }
{\langle H \coma R \coma \IFNULL\Sirx\ \THEN  s_{1}\ \ELSE  s_{2} \coma  n, C\rangle
  \xlongrightarrow{\snnull}_D
  \langle H\coma R\coma s_{2} \coma  n, C\cup \{\snnull\} \rangle}

\rulesp

\infrule[Sem-LetDeref]
{x' \notin \DOM(R) \andalso R(y) \in \DOM(H)}
{\langle H\coma R\coma  \LET x = *y \; \IN s , n, C \rangle
  \longrightarrow_{D}
  \langle H\coma R\Lfc x' \mapsto H(R(y)) \Rfc \coma   \Lb x'/x \Rb s , n, C\rangle }

\rulesp

\infrule[Sem-Free]
{R(x) \neq \NULL \mbox{ and } R(x) \in \DOM(H)}
{\langle H\set{R(x) \mapsto v}\coma R\coma \Free(x) , n, C \rangle \xlongrightarrow{\Free}_{D}
  \langle H\backslash R(x) \coma R \coma \SKIP , n+1, C \rangle}

\rulesp

\infrule[Sem-Malloc]
{l \notin \DOM(H) \andalso n > 0}
{\langle H\coma R\coma  \LET x = \Malloc() \; \IN s , n, C\rangle
  \xlongrightarrow{\Malloc}_{D}
  \langle H \Lfc l \mapsto v\Rfc \coma R\Lfc x' \mapsto l \Rfc \coma   \Lb x'/x \Rb s , n-1, C  \rangle }

\rulesp
\begin{minipage}{0.5\textwidth}
\infrule[Sem-Call]
{D(f) = (\vec{y})s}
{ \langle H\coma R\coma  f(\vec{x}) , n, C\rangle
  \longrightarrow_{D}
  \langle H\coma R\coma  \Lb \vec{x}/\vec{y} \Rb s , n, C \rangle}
\end{minipage}
\begin{minipage}{0.5\textwidth}
\infrule[Sem-FreeExn]
{R(x) = \NULL \mbox{ or } R(x) \notin \DOM(H)}
{\langle H\coma R\coma \Free(x) , n, C\rangle \xlongrightarrow{\Free}_{D} \MEMEX}
\end{minipage}
\rulesp

\begin{minipage}{0.5\textwidth}
\infrule[Sem-AssignExn]
{R(x) = \NULL \mbox{ or } R(x) \notin \DOM(H)}
{\langle H\coma R\coma  *x \leftarrow y , n, C\rangle
  \longrightarrow_{D} \MEMEX }
\end{minipage}
\begin{minipage}{0.5\textwidth}
\infrule[Sem-DerefExn]
{R(y) = \NULL \mbox{ or } R(y) \notin \DOM(H)}
{\langle H\coma R\coma  \LET x = *y \; \IN s, n, C\rangle
    \longrightarrow_{D} \MEMEX}
\end{minipage}
\rulesp

\infrule[Sem-AssignConstExn]
{ \exists z. \scon(*z) \in C \mbox{ and } R(x) = R(z)}
{ \langle H \set{R(x) \mapsto v}, R, *x \leftarrow y , n, C\rangle \xlongrightarrow{}_{D}
  \CONSTEX}

%% \infrule[Sem-ConstExn]
%%         { H(R(x)) \neq H'(R'(x))
%%           \Rtab
%%           \langle H, R, s, n, C \rangle \xlongrightarrow{\rho}_{D} \langle H', R', s', n', C \rangle}
%% {\langle H, R, \scon\Sirx s, n, C \rangle \xlongrightarrow{\rho}_{D} \CONSTEX }

\infax[Sem-OutOfMem]
{ \langle H\coma R\coma \LET x = \Malloc() \ \IN s ,  0, C  \rangle    \xlongrightarrow{\Malloc}_{D}
  \OVERFLOW}

\end{minipage}

\caption{Operational semantics of \(\mathcal{L}\).}
\label{fig:transitionRules}
\end{figure*}

The relation $\xlongrightarrow{\rho}_D$ is the least relation that
satisfies the rules in Figure~\ref{fig:transitionRules}.  In these
rules, $E$ is a \emph{context} defined by $E ::= E;s \mid []$.

The operational semantics is, except for the multiset of facts $C$,
almost the same as the previous framework~\cite{}.  Important invariants
of a configuration is that (1) if $C$ contains $\scon(*x)$, then the
value of the expression $*x$ is not changed in the next step and (2) if
$C$ contains $\snull$ (resp., $\snnull$), then the value of $*x$ is
equal to $\NULL$ (resp., not equal to null).

In order to manipulate the $C$ part of a configuration, we use a
function $\FILTER(C, *x)$ in the rules for the operational semantics
defined as follows:
\[
\begin{array}{l}
 \FILTER(C, *x) =\\
 \quad \LET\ C' = C - \scon\Sirx\ \IN\\
 \quad\quad \IF\ \scon\Sirx \in C'\ \THEN\ C'\\
 \quad\quad \ELSE\ C'\backslash\{\snull,\snnull\}.
\end{array}
\]
The function $\FILTER(C, *x)$ sets $C'$ to the multiset obtained by
removing one $\scon(*x)$ from $C$.  (Recall that $C$ is a
\emph{multiset} of facts.)  If $C'$ does not contain $\scon(*x)$, then
it deletes all the facts related to $*x$.  Otherwise, it returns $C'$.

We explain the rules related to path sensitivity and the
constantness annotations; for the other rules, see~\cite{}.
\begin{itemize}
\item \textsc{Sem-ConstSeq}: The end of the $\mathbf{const}$ block is
      marked at the end of the current continuation.  The fact
      $\scon(*x)$ is also added to $C$ to record that $*x$ is declared
      to be constant until $\Endconst$ is encountered.
\item \textsc{Sem-Assign} and \textsc{Sem-ConstAssignExn}: These rules
      handle assignment statements.  If constantness of $*z$ is declared
      (i.e., $\scon(*z) \in C$), then a write operation to $*z$ leads
      to an exception $\CONSTEX$.  This exception is raised if a
      constantness annotation provided by a programmer is wrong.  The
      type system introduced later does not guarantee unreachability to
      this exceptional state.
\item \textsc{Sem-IfNullT}, \textsc{Sem-IfNullF},
      \textsc{Sem-IfConstNullT}, and \textsc{Sem-IfConstNullF}: These
      rules handle an $\mathbf{if}$-statements depending on whether $*x$
      is null or not.  The semantics records which branch is taken to
      the multiset of facts $C$ only if $C$ contains $\Sirx$.
      Otherwise, even if the semantics recorded $\snull(*x)$ or
      $\snnull(*x)$ to $C$, it may not be reliable in the sense of the
      invariant (2) above since the value of $*x$ may be changed by an
      assignment via an alias of $x$.
\item \textsc{Sem-ConstSkip}: By using the function $\FILTER$ above,
      this rule removes one $\mathbf{endconst}(*x)$ from $C$ and
      conducts ``garbage collection'' of the facts related to $*x$ if
      necessary.
\item \textsc{Sem-Malloc} and \textsc{Sem-OutOfMem}: $\Malloc$ allocates
      a new memory cell by extending the heap $H$ and decreasing the
      value of $n$ by $1$.  If there is no available cell (i.e., $n =
      0$), then $\Malloc$ raises $\OVERFLOW$ exception.
\item \textsc{Sem-Free}: $\Free(x)$ removes the entry for $R(x)$ from
      the heap $H$ and increase the value of $n$ by $1$.
\item \textsc{Sem-AssignExn}, \textsc{Sem-FreeExn} and
      \textsc{Sem-DerefExn}: If a program tries to access to a
      deallocated memory cell, then the exception $\MEMEX$ is raised.
\end{itemize}

\begin{myDef}
\label{df:ml} A program $\langle D, s \rangle$ is said to \emph{execute
within $n$ memory cells} if $\langle \emptyset, \emptyset, s, n,
\emptyset\rangle \not\xlongrightarrow{}_D^* \MEMEX$.
\end{myDef}


% In order to deal with a path-sensitive program to guarantee
% \emph{total} memory-leak freedom, we redefined the several definitions
% as follows.  defined as follows:
% \begin{myDef}[total memory-leak freedom]
% \label{df:ml}
% A program \(\langle D, s \rangle\) is \emph{totally memory-leak free}
% if there is a natural number \(n\) such that it does not require more
% than \(n\) cells.
% \end{myDef}
% \begin{myDef}[Memory leak]
% \label{df:ml}
% A configuration \(\langle H, R, s, n, C \rangle\) \emph{goes overflow} if
% there is \(\sigma\) such that \(\langle H, R, s, n, C \rangle
% \xLongrightarrow{\sigma} \OVERFLOW\).  A program \(\langle D, s
% \rangle\) \emph{consumes at least \(n\) cells} if \(\langle \emptyset,
% \emptyset, s, n, \emptyset \rangle\) goes overflow.  
% \end{myDef}



\section{Type system}
\label{sec:typesystem}

\begin{figure*}
\[
\begin{array}{rlcl}
  P & (\mbox{behavioral types})&::=& \mathbf{0} \mid P_{1};P_{2} \mid \Malloc \mid \Free \mid (x)P \mid (*x)(P_1,P_2) \mid \scon\Sirx P  \mid \Endconst\\\\
  &  &  & \mid \alpha \mid \mu\alpha.P\\
  \Gamma & (\mbox{variable type environment}) &::=& \set{x_1, x_2, \dots, x_n}\\
  \Psi & (\mbox{dependent function type}) &::=& (\vec{x})P\\
  \Theta & (\mbox{function type environment}) &::=& \set{f_1\COL \Psi_1,\dots,f_n\COL \Psi_n}\\
  % k & (\mbox{facts}) &::=& \snull \mid \snnull \mid \scon\Sirx   \\
  % F & (\mbox{fact sets}) &::=& \{k_1,...,k_n\} \\
\end{array}
\]
\caption{Syntax of types.} 
\label{fig:typeSyntax}
\end{figure*}

\subsection{Types}

Figure~\ref{fig:typeSyntax} defines the syntax of \emph{types}.
Behavioral types, ranged over by \(P\), specify the behavior of
statements.  The type $\mathbf{0}$ represents the do-nothing behavior.
The type $P_1;P_2$ is the sequential composition of $P_1$ and $P_2$.
The types $\Malloc$ and $\Free$ represents one allocation and one
deallocation, respectively.  The type $(x)P$ is the behavior that binds
a variable $x$ and behaves as specified by $P$; the type $P$ may be
dependent to $x$.  For example, the behavior of the statement $\LET\ x =
y\ \IN\ \SKIP$ is represented by $(x)\mathbf{0}$ since it binds the
variable $x$ and does nothing. The type $(*x)(P_1,P_2)$ is the behavior
of a branching statement.  This type represents the behavior $P_1$ if
$*x$ is null; $P_2$ otherwise.  The type represents the behavior of a
statement that works as specified in $P$ under the assumption that the
value of $*x$ does not change.  The type $\Endconst$ represents the
behavior that marks the end of the constantness annotation.  The types
$\alpha$ and $\mu\alpha.P$ allows to specify a recursive behavior.  The
type $\mu\alpha.P$ represents the behavior $P$, in which $\alpha$
represents the behavior of $\mu\alpha.P$ itself.  For example,
$\mu\alpha.(\Malloc;\Free;\alpha)$ represents a statement that repeats
an allocation followed by a deallocation forever.  $\Gamma$, a
\emph{type environment}, is a set of variables representing the set of
free variables that may appear in a statement.

The types for procedures, ranged over by $\Psi$, is of the form
$(\vec{x})P$.  This type represents a procedure that takes $\vec{x}$ as
arguments and behaves as specified by $P$, which may be dependent to
$\vec{x}$.  Procedure type environments, ranged over by $\Theta$,
assigns a procedure type to a function name.

% \(k\) represents constant values information, where \(\snull\) represents
% \(\Sirx\) is a null pointer; \(\snnull\) represents \(\Sirx\) is not a
% null pointer; \(\scon\Sirx\) represents \(\Sirx\) should be a constant.

% Constant value environment ranged over by \(F\) is a set of constant
% values information.

% \paragraph{Notation}
% \(filter\_T(F, *x)\) is defined by a pseudcode as follows:
% \[
% \begin{array}{lcl}
%   filter\_T(F, *x) &=& let \; F' = F - \scon\Sirx \; in \\
%   & & if \; \scon\Sirx \notin \; F'\; then \; return \; (F' \backslash \{\snull,\snnull\})\\
%   & & else \; return \; F'
% \end{array}
% \]

Figure~\ref{fig:bdRules} defines the operational semantics of the
behavioral types.  The semantics is defined by a labeled transition
system over configuration of the shape $\langle P, C \rangle$.  Here,
$C$ is the multiset of facts that holds.  We add explanation to
important rules.
\begin{itemize}
 \item \textsc{Tr-Bind}: The behavioral type $(x)P$ picks up a fresh
       name $x'$ and makes transition to $[x'/x]P$.  To understand this
       definition, recall that the rules in
       Figure~\ref{fig:transitionRules} for dealing with statements with
       a binder (i.e., \textsc{Sem-LetNull}, \textsc{Sem-LetEq},
       \textsc{Sem-LetDeref}, and \textsc{Sem-Malloc}) generate a fresh
       variable for the bound variable and add it to the environment $R$
       of the configuration.  The rule \textsc{Tr-Bind} simulates this
       behavior.
 \item \textsc{Tr-Const} and \textsc{Tr-EndConst}: The behavioral type
       $\scon(*x)P$ adds the fact $\scon(*x)$ to $C$ and evolves to
       $P;\Endconst$.  If $\Endconst$ is encountered, then all the facts
       related to $*x$ in $C$ is removed by $\FILTER$.  This is parallel
       to the semantics of $\scon(*x)s$ and $\Endconst$ in
       Figure~\ref{fig:transitionRules}.
 \item \textsc{Tr-ConstNondet1} and \textsc{Tr-ConstNondet2}: If the
       fact $\scon(*x)$ is a member of $C$ and there is no $\NULL(*x)$
       nor $\neg\NULL(*x)$, then the behavioral type $(*x)(P_1,P_2)$
       nondeterministically chooses $P_1$ or $P_2$.  The fact
       $\NULL(*x)$ (resp. $\neg\NULL(*x)$) is added to $C$ if the branch
       $P_1$ (resp. $P_2$) is taken.  This addition of the fact is
       conducted only if $\scon(*x)$ is a member of $C$ (cf.,
       \textsc{Tr-NoConst1} and \textsc{Tr-NoConst2}).
 \item \textsc{Tr-ConstNull} and \textsc{Tr-ConstNotNull}: If
       $\scon(*x)$ is a member of $C$ and if $\NULL(*x)$ (resp.
       $\neg\NULL(*x)$) is a member of $C$, then the behavioral type
       $(*x)(P_1,P_2)$ evolves to $P_1$ (resp. $P_2$).  This happens
       only if (1) the behavioral type $(*x)(P_1,P_2)$ was inside the
       block of certain $\scon(*x)\{\dots\}$ and (2) there was another
       $(*x)(P_1',P_2')$ was in the same block before $(*x)(P_1,P_2)$ is
       reached.  Because the transition of $(*x)(P_1',P_2')$ should have
       registered $\NULL(*x)$ or $\neg\NULL(*x)$ to $C$ and both
       $(*x)(P_1',P_2')$ and $(*x)(P_1,P_2)$ are inside
       $\scon(*x)\{\dots\}$, it is safe to assume that $(*x)(P_1,P_2)$
       takes the same branch as $(*x)(P_1',P_2')$.
 \item \textsc{Tr-NoConst1} and \textsc{Tr-NoConst2}: If $\scon(*x)$ is
       not in the multiset of facts $C$, then the behavioral type
       $(*x)(P_1,P_2)$ nondeterministically evolves to $P_1$ or $P_2$.
\end{itemize}

% Figure~\ref{fig:bdRules} depicts semantics of behavioral types with
% dependent types, and they are given by the labeled transition
% system. The relation \( \langle P, F \rangle \xlongrightarrow{\rho}
% \langle P', F' \rangle \) means that \(P\) can make an action \(\rho\),
% and \(P\) turns into \(P'\) after it makes action \(\rho\); \(F\) and
% \(F'\) record constant value environment before and after making action
% \(\rho\) respectively.


\begin{figure*}
 \begin{minipage}{\textwidth}


\infax[Tr-Skip]
{ \langle \mathbf{0};P, C \rangle \rightarrow \langle P, C \rangle }

\infrule[Tr-Seq]
{ \langle P_1, C \rangle \xlongrightarrow{\rho} \langle P_1', C' \rangle }
{ \langle P_1;P_2, C \rangle \xlongrightarrow{\rho} \langle P_1';P_2, C' \rangle }

%% \begin{minipage}{0.5\textwidth}
%% \infax[Tr-Malloc]
%%  { \langle \Malloc, F \rangle \xlongrightarrow{\Malloc} \langle {\bf 0}, F \rangle }
%% \end{minipage}

\infax[Tr-Malloc]
{ \langle \Malloc, C \rangle \xlongrightarrow{\Malloc} \langle \mathbf{0}, C \rangle }

\infax[Tr-Free]
{ \langle \Free, C \rangle \xlongrightarrow{\Free} \langle \mathbf{0}, C \rangle }

\infrule[Tr-Bind]
{\mbox{$x'$ is fresh.}}
{ \langle (x)P, C \rangle \rightarrow \langle [x'/x]P, C \rangle }

\infax[Tr-Const]
{ \langle \scon\Sirx P, C \rangle \rightarrow \langle P;\Endconst, C + \{\scon\Sirx\} \rangle }

\infrule[Tr-Endconst]
{C' = \FILTER(C, *x)}
{ \langle \Endconst, C \rangle \rightarrow \langle \mathbf{0}, C' \rangle }

\begin{minipage}{0.5\textwidth}
\infrule[Tr-NoConst1]
{ \scon\Sirx \notin C }
{ \langle \Sirx (P_1,P_2), C \rangle \xlongrightarrow{\snull} \langle P_1, C \rangle }
\end{minipage}
\begin{minipage}{0.5\textwidth}
\infrule[Tr-NoConst2]
{ \scon\Sirx \notin C }
{ \langle \Sirx (P_1,P_2), C \rangle \xlongrightarrow{\snnull} \langle P_2, C \rangle }
\end{minipage}
%% \infrule[Tr-NullNotIn]
%% { \snull \notin F \andalso \scon\Sirx \in F }
%% { \langle \Sirx (P_1,P_2), F \rangle \rightarrow \langle P_2, F\cup\{\snnull\} \rangle }

\begin{minipage}{0.5\textwidth}
\infrule[Tr-ConstNull]
{ \snull, \scon\Sirx \in C }
{ \langle \Sirx (P_1,P_2), C \rangle \rightarrow \langle P_1, C \rangle }
\end{minipage}
\begin{minipage}{0.5\textwidth}
\infrule[Tr-ConstNotNull]
{ \snnull, \scon\Sirx \in C }
{ \langle \Sirx (P_1,P_2), C \rangle \rightarrow \langle P_2, C \rangle }
\end{minipage}


\infrule[Tr-ConstNondet1]
{ \snull, \snnull \notin C \andalso \scon\Sirx \in C }
{ \langle \Sirx (P_1,P_2), C \rangle \xlongrightarrow{\snull} \langle P_1, C\cup{\snull} \rangle }

\infrule[Tr-ConstNondet2]
{ \snull, \snnull \notin C \andalso \scon\Sirx \in C }
{ \langle \Sirx (P_1,P_2), C \rangle \xlongrightarrow{\snnull} \langle P_2, C\cup{\snnull} \rangle }

% \begin{minipage}{0.5\textwidth}
% \infax[Tr-ChoiceL]
% { \langle P_1 + P_2, F \rangle \rightarrow \langle P_1, F \rangle }
% \end{minipage}
% \begin{minipage}{0.5\textwidth}
% \infax[Tr-ChoiceR]
% { \langle P_1 + P_2, F \rangle \rightarrow \langle P_2, F \rangle }
% \end{minipage}

\infax[Tr-Rec]
 { \langle \mu\alpha.P, C \rangle \rightarrow \langle
   [\mu\alpha.P/\alpha]P, C \rangle }

% \infax[Tr-LetX*Y]
% { \langle \LET x = *y \; \IN P, F \rangle \rightarrow \langle [x'/x]P, F \rangle }

% \infax[Tr-LetX]
% { \langle \LET x = \NULL \; \IN P, F \rangle \rightarrow \langle [x'/x]P, F \rangle }

%% \infrule[Tr-NNullNotIn]
%% { \snnull \notin F \andalso \scon\Sirx \in F}
%% { \langle \Sirx (P_1,P_2), F \rangle \rightarrow \langle P_1, F\cup\{\snull\} \rangle }
 
%%   \label{df:fv}
%%   \mbox{The set of free variables of behavioral type \(P\) is defined as
%%     follows:}

%%   \( \mathbf{FV}(\bf{0}) = \mathbf{FV}(\Malloc) =
%%   \mathbf{FV}(\Free) = \mathbf{FV}(\alpha) = \emptyset \) 

%%   \( \mathbf{FV}(P_1 + P_2) = \mathbf{FV}(P_1;P_2) = \mathbf{FV}(P_1) \cup \mathbf{FV}(P_2)\)

%%   \( \mathbf{FV}((x)(P_1,P_2)) \) = \{x\} \(\cup \ \mathbf{FV}(P_1) \cup
%%   \mathbf{FV}(P_2)\)
%% \end{myDef}
\end{minipage}
\caption{semantics of behavioral types with dependent types.}
\label{fig:bdRules}
\end{figure*}

\begin{exmp}
Let $P$ be the behavioral type
\[
 \scon(*x)\{(*x)(\Malloc,\mathbf{0});(*x)(\Free,\mathbf{0})\}.
\]
We write $P_1$ for $(*x)(\Malloc,\mathbf{0})$, $P_2$ for
$(*x)(\Free,\mathbf{0})$, $C_1$ for $\set{\scon(*x)}$, $C_2$ for
$\set{\NULL(*x)}$, and $C_3$ for $\set{\neg\NULL(*x)}$.  Then, the
following transition sequence is possible.
\[
 \begin{array}{ll}
                 & \langle P, \emptyset \rangle\\
  \xrightarrow{} & \langle P_1;P_2;\Endconst, C_1 \rangle\\
  \xrightarrow{} & \langle \Malloc;P_2;\Endconst, C_1 \cup C_2 \rangle\\
  \xrightarrow{\Malloc} & \underline{\langle P_2;\Endconst, C_1 \cup C_2 \rangle}\\
  \xrightarrow{} & \langle \Free;\Endconst, C_1 \cup C_2 \rangle\\
  \xrightarrow{\Free} & \langle \Endconst, C_1 \cup C_2 \rangle\\
  \xrightarrow{} & \langle \mathbf{0}, \emptyset \rangle.\\
 \end{array}
\]
Notice that, from the underlined configuration, the transition to
$\langle \mathbf{0};\Endconst, C_1 \cup C_2 \rangle$ is not allowed
because the facts include $\scon(*x)$ and $\NULL(*x)$.  Another possible
transition is
\[
 \begin{array}{ll}
                 & \langle P, \emptyset \rangle\\
  \xrightarrow{} & \langle P_1;P_2;\Endconst, C_1 \rangle\\
  \xrightarrow{} & \langle \mathbf{0};P_2;\Endconst, C_1 \cup C_3 \rangle\\
  \xrightarrow{} & \underline{\langle P_2;\Endconst, C_1 \cup C_3 \rangle}\\
  \xrightarrow{} & \langle \mathbf{0};\Endconst, C_1 \cup C_3 \rangle\\
  \xrightarrow{} & \langle \Endconst, C_1 \cup C_3 \rangle\\
  \xrightarrow{} & \langle \mathbf{0}, \emptyset \rangle.\\
 \end{array}
\]
A transition to $\langle \Free;\Endconst, C_1 \cup C_3 \rangle$ is not
allowed because $C_1 \cup C_3$ contains $\scon(*x)$ and $\neg\NULL(*x)$.
Notice that the path-sensitive behavioral type excludes the possibility
of $\langle P, \emptyset \rangle \xrightarrow{\Malloc} \langle
\mathbf{0}, \emptyset \rangle$ and $\langle P, \emptyset \rangle
\xrightarrow{\Free} \langle \mathbf{0}, \emptyset \rangle$.
\end{exmp}

\subsection{Type judgment}

The type judgment for statements is of the form \(\Theta; \Gamma \vdash
s : P \).  The judgment reads ``the behavior of $s$ is $P$ under the
function type environment $\Theta$ and the variable type environment
$\Gamma$''.

We incorporate subtyping to our type system defined as follows.
\begin{myDef}[Subtyping]
\label{df:subtype} \(C \vdash P_1 \le P_2\) is the largest relation such
that, for any \(P_1'\), $C'$, and \(\rho\), if \( \langle P_1, C \rangle
\xlongrightarrow{\rho} \langle P_1', C' \rangle \), then there exists
\(P_2'\) such that \( \langle P_2, C \rangle \xLongrightarrow{\rho}
\langle P_2', C' \rangle \) and \(C' \vdash P_1' \le P_2'\).  We write
\(P_1 \le P_2\) if \(C \vdash P_1 \le P_2\) for any F.
\end{myDef}

\begin{figure*}
\begin{minipage}{\textwidth}

\begin{minipage}{0.5\textwidth}
\infax[T-Skip]
{\Theta ; \Gamma \vdash \SKIP : \mathbf{0}}
\end{minipage}
\begin{minipage}{0.5\textwidth}
\infrule[T-Seq]
{\Theta ; \Gamma \vdash s_{1} : P_{1} \andalso \Theta ; \Gamma \vdash s_{2} : P_{2}}
{\Theta ; \Gamma \vdash s_{1} ; s_{2} : P_{1};P_{2} }
\end{minipage}


\begin{minipage}{0.5\textwidth}
\infax[T-Assign]
{\Theta ; \Gamma, x, y \vdash *x \leftarrow y : \mathbf{0} }
\end{minipage}
\begin{minipage}{0.5\textwidth}
\infax[T-Free]
{\Theta ; \Gamma, x \vdash \Free(x) : \Free}
\end{minipage}


\begin{minipage}{0.5\textwidth}
\infrule[T-Malloc]
{\Theta ; \Gamma,x \vdash s : P}
{\Theta ; \Gamma \vdash \LET x = \MALLOC\ \IN s \COL \Malloc;(x)P}
\end{minipage}
\begin{minipage}{0.5\textwidth}
\infrule[T-LetEq]
{\Theta ; \Gamma , x, y  \vdash s : P}
{\Theta ; \Gamma, y \vdash \LET x = y\ \IN s : [y/x]P}
\end{minipage}


\begin{minipage}{0.5\textwidth}
\infrule[T-LetDeref]
{\Theta ; \Gamma , x, y  \vdash s : P}
{\Theta ; \Gamma, y \vdash \LET x = *y\ \IN s : (x)P}
\end{minipage}
\begin{minipage}{0.5\textwidth}
\infrule[T-LetNull]
{\Theta ; \Gamma, x  \vdash s : P  \Rtab}
{\Theta ; \Gamma \vdash \LET x = \NULL\ \IN s : (x)P}
\end{minipage}

\infax[T-Endconst] 
{\Theta ; \Gamma,x \vdash \Endconst : \Endconst}

\infrule[T-Const] 
{\Theta ; \Gamma, x \vdash s : P }
{\Theta ; \Gamma,x \vdash \scon\Sirx s : \scon\Sirx P}


%%\begin{minipage}{0.5\textwidth}  \andalso x \in \mathbf{FV}(P_1) \cup \mathbf{FV}(P_2)
\infrule[T-IfNull] 
{\Theta ; \Gamma, x \vdash s_{1} : P_1 \andalso \Theta ; \Gamma, x \vdash s_{2} : P_2}
{\Theta ; \Gamma, x \vdash \IFNULL\Sirx \; \THEN s_{1}\; \ELSE s_{2} : (*x)(P_1,P_2)}
%%\end{minipage}
%%\begin{minipage}{0.5\textwidth}
\infax[T-Call] {\Theta, f \COL (\vec{x})P; \Gamma, \vec{y} \vdash f(\vec{y}) :  [\vec{y}/\vec{x}]P}
%%\end{minipage}

%%\vspace{2mm} 

\infrule[T-Sub]
{\Theta ; \Gamma \vdash s : P_{1} \andalso P_{1} \le P_{2}}
{\Theta ; \Gamma \vdash s : P_{2}}

\infrule[T-Def] { \Theta(f) = (\vec{x})P \andalso \DOM(D) = \DOM(\Theta)
        \andalso \Theta; x_1,\dots,x_n \vdash s \COL P \mbox{ for each
        \(f \mapsto (x_1,\dots,x_n)s \in D\)} } {\vdash D \COL \Theta}
% Program
\infrule[T-Program]
{\vdash D : \Theta \andalso \Theta; \emptyset\vdash s : P}
{\vdash \langle D, s \rangle : P}

\end{minipage}
\caption{Typing rules.}
\label{fig:TypingRules}
\end{figure*}

The type judgment $\Theta; \Gamma \vdash s \COL P$ is defined as the
least relation that satisfies the rules in Figure~\ref{fig:TypingRules}.
We give explanation to several important rules.
\begin{itemize}
 \item \textsc{T-Malloc}: The statement $\LET\ x = \Malloc()\ \IN\ s$
       performs an allocation once, binds the variable $x$, and then
       executes $s$.  The behavior of the whole statement is therefore
       $\Malloc;(x)P$.
 \item \textsc{T-LetEq}: The statement $\LET\ x = y\ \IN\ s$ makes an
       alias $x$ of $y$ and then executes $s$.  Therefore, we rename $x$
       to $y$ in the behavioral type $P$ of $s$.
 \item \textsc{T-LetDeref} and \textsc{T-LetNull}: The statements $\LET\
       x = *y\ \IN s$ and $\LET\ x = \NULL\ \IN\ s$ bind $x$ and then
       execute $s$.  Therefore, the type of the whole statement is
       $(x)P$.  The type system just ignores that what the variable $x$
       is bound to.
 \item \textsc{T-IfNull}: The statement 
       \[
       \IFNULL(*x)\ \THEN\ s_1\ \ELSE\ s_2
       \]
       branches depending on whether $*x$ is null or not.  This behavior
       is abstracted by the type $(*x)(P_1,P_2)$.
 \item \textsc{T-Call}: If the type of the procedure $f$ is
       $(\vec{x})P$, then the type of $f(\vec{y})$ is
       $[\vec{y}/\vec{x}]P$; notice that $P$ may depend on the arguments
       of the procedure.
\end{itemize}

Typing rules for the declarations $D$ and programs $\langle D, s
\rangle$ are also shown in Figure~\ref{fig:TypingRules}.  These rules
are standard.

\subsection{Type soundness}

Although we have not completed the proof concretely, we conjecture that
the following preservation theorem for the judgment $\Theta;\Gamma
\vdash s \COL P$ holds.
\begin{conjecture}[Preservation]
 \label{conj:preservation}
 Suppose that
 \begin{itemize}
  \item $\vdash D \COL \Theta$,
  \item $\Theta;\Gamma \vdash s \COL P$,
  \item $\Gamma \subseteq \DOM(R)$,
  \item $\langle H,R,s,n,C \rangle \xlongrightarrow{\rho} \langle H',R',s',n',C' \rangle$,
 \end{itemize}
 Then, there exist $\Gamma'$ and $P'$ such that
 \begin{itemize}
  \item $\Theta;\Gamma' \vdash s' \COL P'$,
  \item $\Gamma' \subseteq \DOM(R')$,
  \item $\langle P,C \rangle \xLongrightarrow{\rho} \langle P',C' \rangle$.
 \end{itemize}
\end{conjecture}

% \begin{lemma}[Preservation]
% \label{lem:preservation}
% suppose that \( \Theta; \Gamma \vdash \langle H, R, s, n, C \rangle :
% \langle P, F \rangle\). If \( \langle H, R, s, n, C \rangle
% \xlongrightarrow{\rho} \langle H', R', s', n', C' \rangle\) then
% \(\exists P', F'\) s.t. (1) \( \Theta; \Gamma \vdash \langle H', R',
% s', n', C' \rangle : \langle P', F' \rangle\) and (2) \(\langle P, F
% \rangle \xLongrightarrow{\rho} \langle P', F'\rangle \).
% \end{lemma}

The following fact is a corollary of Conjecture~\ref{conj:preservation}.
\begin{conjecture}\label{cor:progAbstract}
 If $\vdash \langle D, s \rangle : P$ and $\langle \emptyset, \emptyset,
 s, n, \emptyset \rangle \xLongrightarrow{\sigma}_D$, then $\langle P,
 \emptyset \rangle \xLongrightarrow{\sigma}$.
\end{conjecture}

\subsection{Verification of memory-leak freedom of a program using 
the inferred behavioral type}

Conjecture~\ref{cor:progAbstract} states that the behavior of a program
is soundly abstracted by its type.  This section describes how the
inferred type can be used to estimate an upper bound of memory
consumption.  We first define ``memory consumption'' of a behavioral
type.

\begin{myDef}
\label{df:okn} We write \(OK_n(P, C)\) if (1) \(\langle P, F \rangle
\xlongrightarrow{\sigma} \langle P', F'\rangle \) implies
\(\sharp_\Malloc(\sigma) - \sharp_\Free(\sigma) \le n\) where
$\sharp_\rho(\sigma)$ is the number of $\rho$ in $\sigma$.
\end{myDef}

\begin{conjecture}
\label{lem:immediateSafety}
If
 \begin{itemize}
  \item $\Theta;\Gamma \vdash s \COL P$,
  \item $\Gamma \subseteq \DOM(R)$,
  \item $\OK_n(P,C)$, and
  \item $n' \ge n$,
 \end{itemize}
 Then, \(\langle H, R, s, n, C \rangle \xLongrightarrow{\sigma}^* \OVERFLOW \).
\end{conjecture}

The conjecture above states that, if $\vdash \langle D, s \rangle \COL
P$ and $\OK_n(P,\emptyset)$ hold, then $n$ is an upper bound of the
program.  Therefore, by conducting type inference to the program,
obtaining a type, say, $P$, and estimating an upper bound of $P$, we can
estimate an upper bound of the memory consumption of the program.

% FROM HERE.

% \begin{myDef}
% \label{df:okf} \(OK(F)\) holds if F does not contain both \( \snull \)
% and \( \snnull \).
% \end{myDef}

% \begin{myDef}
%   \label{df:Rdu}
%  we write \( \Theta; \Gamma \vdash \langle H, R, s, n, C \rangle : \langle P,
%   F \rangle\), if \(\Theta; \Gamma \vdash s : P\) and \(OK_n(P, F)\) with \(C \thickapprox F\).  
% \end{myDef}

% The proof is based on the following lemmas: preservation and lack of
% immediate overflow.

% %% \begin{lemma}[Preservation]
% %% \label{lem:preservation}
% %% If \(OK_{n}(P, F)\), \(\Theta; \Gamma \vdash s : P \), \(\vdash D \COL
% %% \Theta\), and \( \langle H, R, s, n, C \rangle \xlongrightarrow{\rho}
% %% \langle H, R', s', n', C' \rangle \), then there exists \(P'\) and \(F'\)
% %% such that (1) \( \Theta; \Gamma^{'} \vdash s' : P'\), (2)
% %% Rab(\(\langle P, F \rangle \xlongrightarrow{\rho} \langle P', F'
% %% \rangle\)), and (3) \(OK_{n'}(P', F')\).
% %% \end{lemma}



% %% \begin{lemma}[Lack of immediate overflow]
% %% \label{lem:immediateSafety}
% %% If \(\Theta; \Gamma \vdash s \COL P \), \(\vdash D \COL \Theta\), and
% %% \(\OK_n(P)\), then \(\langle H, R, s, n \rangle
% %% \not\xlongrightarrow{\Malloc} \OVERFLOW \).
% %% \end{lemma}

% Before showing typing rules for statements in
% Figure~\ref{fig:TypingRules}, we need explain several important
% definitions. The first one is \(OK_n(P, F)\), a predicate, where \(P\)
% represents the behavior of a program which consumes at most \(n\) memory
% cells under constant value environment \(F\).

% \begin{myDef}[\(\sharp_{\rho}(\sigma)\)]
% \label{df:sharf}
% \(\sharp_{\rho}(\sigma)\) is the number of \(\rho\) in the sequence
% \(\sigma\).
% \end{myDef}

% %% \begin{myDef}
% %% \label{df:const}
% %% \(\sconst\) means:\\
% %% \( \forall \sigma',\sigma''\). \(\sigma'\) is a subsequence
% %% of \(\sigma\), and \(\sigma' = \Startconst;\sigma'';\Endconst\) or \(\sigma' = \sigma'';\Endconst\). The \(\sigma''\) does not contain \(\Startconst\) or \(\Endconst \), and it also does not contain both \(\sassx\) and \(\sassxn\).
% %% \end{myDef}

% Intuitively, \(OK_n(P, F)\) represents at very running steps, the
% number of memory cells a program consumed will not exceed the number
% of memory cells the program requires.

% Figure~\ref{fig:TypingRules} shows the typing rules. For example, the
% rule \rn{T-IfNull} represents the behavior of \(\IFNULL\Sirx \; \THEN
% s_{1}\; \ELSE s_{2}\) is abstracted as \(\Sirx(P_1, P_2)\) where
% \(P_1\) and \(P_2\) are the behavior of \(s_1\) and \(s_2\)
% respectively; this conditional statement means that executing \(s_1\)
% if \(\Sirx\) is a null pointer, otherwise \(s_2\).  The typing rule
% \rn{T-Program} represents a program requires at most \(n\) memory
% cells during running under the predication \(OK_n(P, F)\), where \(P\)
% is behavioral type of statement \(s\).



\section{Type Reconstruction}
\label{sec:reconstruction}
In this section we proposed a constraint-based type reconstruction
procedure for the extended type system, which is similiar to the one
in Kobayashi et al.~\cite{DBLP:journals/lmcs/KobayashiSW06}.  This
procedure can generate constraints for a given program to be definite
and well-typed by constructing a derivation tree based on the rules in
Figure~\ref{fig:TypingRules}. According to this procedure, the
generated constraint is either a subtyping constraint \(P \le \alpha
\) or \(\OK_\nu(\alpha, F)\) in which \(\nu\) represents an unknown
number of available memory cells and \(F\) represents constan value
environment. The predication \(OK_n(P, F)\) only appears in the rule
\(\textsc{T-Program}\), therefore only one constraint of
\(OK_\nu(\alpha, F)\) is included in the constraints set.

To solve the constraint \(OK_\nu(\alpha, F)\), we used several
definitions proposed by Kobayashi et al.~\cite[Lemma
  3.8]{DBLP:journals/lmcs/KobayashiSW06} which describe that a
subtyping constraint \(\alpha \ge P\) can be resolved by setting
\(\alpha = \mu \alpha. P\).  Therefore, the generated constraints set
can be reduced to a single constraint \(\OK_{\nu}(P', F)\) for some
behavioral type \(P'\).

By definition, \(\OK_{\nu}(P, F)\) holds if (1) for any \(\sigma\) and
\(P'\), if \(\langle P, F \rangle \xlongrightarrow{\sigma} \langle P',
F'\rangle \), then there exists a natural number \(n\) such that
\(\sharp_{\Malloc}(\sigma) - \sharp_{\Free}(\sigma) \le n\) and (2)
\(OK(F)\). To prove this predication, we first fix an upper bound for
\(\nu\) and encoding \(F\) by hand.  Then, \(\OK_{\nu}(P, F)\) can be
checked, by using some mode checkers like
CPAChecker~\cite{beyer2011cpachecker}, in finitely many states.


\section{Related Work}\label{sec:relatedwork}

Behavioral types are widely used in program analysis; a survey of this
area can be found in~\cite{DBLP:journals/csur/HuttelLVCCDMPRT16}.
Several representative uses of the behavioral types are ensuring
correctness of the communications conducted among several
processes~\cite{DBLP:journals/corr/abs-1208-6483,DBLP:conf/popl/HondaYC08,DBLP:journals/tcs/CairesV10,DBLP:journals/tcs/IgarashiK04},
static deadlock-freedom
verification~\cite{DBLP:conf/concur/Kobayashi06,DBLP:journals/acta/Kobayashi05}
and static correctness verification of resource-usage patterns (e.g., a
file hanlder is closed before
termination)~\cite{DBLP:journals/lmcs/KobayashiSW06,DBLP:journals/toplas/IgarashiK05}.
Our current work also can be seen as a static analysis of resource-usage
patterns in which memory is a single and unique resource that is
accessed via the primitives $\Malloc$ and $\Free$.

Static verification of memory-leak freedom has been another interesting
topic of program
verification~\cite{DBLP:conf/aplas/SuenagaK09,DBLP:conf/pldi/HeineL03,DBLP:conf/sigsoft/XieA05,DBLP:journals/scp/SwamyHMGJ06,DBLP:conf/sas/OrlovichR06,DBLP:conf/issta/SuiYX12}.
The main interest of the previous work is guaranteeing the following
property: A pointer to every allocated memory cell will be passed to
$\Free$ eventually.  Our work focuses focuses on the sequence of
$\Malloc$ and $\Free$ that may be conducted by a program, forgetting
about which $\Free$ deallocates which memory cell.

% Memory usage is a crucial issue for real-world program, so lots of
% static verification method for memory usage have been
% proposed~\cite{}. However,
% these static methods only guaranteed partial memory-leak freedom and
% lack of illegal accessing of some pointers like null pointers or
% dangling pointers.  In our previous work~\cite{}, we proposed a
% behavioral type system, inspired by Kobayashi et
% al.~\cite{DBLP:journals/lmcs/KobayashiSW06} which guarantees safety
% properties of resources usage for concurrent programs, to estimate the
% upper bound number of memory cells a program consumes.  By using our
% behavioral type system with other static methods mentioned above, we can
% guarantee memory-leak freedom even for nonterminating programs. But
% verification failed for path-sensitive programs, therefore we need to
% assign more information on behavioral type to deal with this
% problem. Our idea is to extend our previous type system with dependent
% types.

% The dependent type~\cite{DBLP:conf/popl/XiP99,DBLP:conf/pldi/XiP98}
% takes more precise information than traditional type, and it can contain
% any values in types and appear as arguments and results of
% functions~\cite{DBLP:conf/tldi/Norell09}. By using this

\section{Conclusion}
\label{sec:conclusion}

We presented a path-sensitive behavioral type system for an imperative
language with manual memory management.  Our type system abstracts the
behavior of a program concerning the primitives $\Malloc$ and $\Free$
for manual memory management.  We stated several conjectures about the
type system.  We also described how the inferred behavioral types can be
used for estimating an upper bound of memory consumption.

We are currently designing and implementing a type inference algorithm
for our type system.  We can infer a type of a program using essentially
the same algorithm as that of Kobayashi et
al.~\cite{DBLP:journals/lmcs/KobayashiSW06}.  We need to check the
feasibility of our type system using practical programs.

The current type system requires a programmer to write constantness
annotation $\scon(*x)$.  The correctness of these annotations are
currently the responsibility of the programmer: If the annotation is
wrong, then the type system does not guarantee anything.  We suppose
that we can verify the correctness of these annotations using our
previous work on \emph{fractional
ownerships}~\cite{DBLP:conf/aplas/SuenagaK09}.  We also expect that we
can automatically insert constantness annotations using the type
inference algorithm of our previous work.


% In order to deal with path-sensitive problem, which results in an
% imprecise abstraction even such that verification failed even for a
% memory-leak free program, we proposed an extension of the previous
% type system with dependent types.  We also described a type
% reconstruction algorithm for this extended type system, and we
% conducted several experiments to prove whether our idea can deal with
% path-sensitivity problem or not.

% Our extended type system can deal with path-sensitivity but only for
% the guard-part of a conditional is a pointer. Therefore verification
% failed on a program where guard-part of a conditional is not a
% pointer. Besides, for simplification our extended type system excludes
% several features of real-world programs. For example, alias pointers
% and variable-sized memory blocks. Encoding the part where a pointer is
% a constant one by hand is unrealistic; Our types ignore the size of
% the allocated block and our system only counts the number of types
% \(\Malloc\) and \(\Free\). Therefore, a program, which contains memory
% leaks by allocating huge memory blocks, may seem to be a well-typed
% one in our type system.  We need to refine our current type system to
% solve these problms.

% In order to solve the constraint of form \(\OK_\nu(P, F)\), we fix an
% upper bound for \(\nu\) at first, which makes our reconstruction
% incomplete. For example, a given program consumes at most \(n\)
% numbers of memory cells, but if the \(\nu\) we chose is great than
% \(n\), the verification holds; otherwise, if the \(\nu\) is less than
% \(n\), the verification failed. The reason is that we have not yet
% known whether there exits an \(n\) s.t. \(OK_n(P, F)\).



%% {\bf 謝辞}\
%% 本論文の初期の版について議論していただいた.....に感謝する.

\bibliographystyle {jssst}
\bibliography {tan}

\appendix
\section{付録: \LaTeX による論文作成のガイド} 

ここに,以前の \verb|sample.tex| では,論文作成のガイドがあったが,
その内容は \verb|guide.tex| に移動した.
\verb|guide.tex| は,スタイルファイル配布物一式の中に含まれている.

\end{document}
