\section{Type system}
\label{sec:typesystem}

\begin{figure*}
\[
\begin{array}{rlcl}
  P & (\mbox{behavioral types})&::=& \mathbf{0} \mid P_{1};P_{2} \mid \Malloc \mid \Free \mid (x)P \mid (*x)(P_1,P_2) \mid \scon\Sirx P  \mid \Endconst\\
  &  &  & \mid \alpha \mid \mu\alpha.P\\
  \Gamma & (\mbox{variable type environment}) &::=& \set{x_1, x_2, \dots, x_n}\\
  \Psi & (\mbox{dependent function type}) &::=& (\vec{x})P\\
  \Theta & (\mbox{function type environment}) &::=& \set{f_1\COL \Psi_1,\dots,f_n\COL \Psi_n}\\
  % k & (\mbox{facts}) &::=& \snull \mid \snnull \mid \scon\Sirx   \\
  % F & (\mbox{fact sets}) &::=& \{k_1,...,k_n\} \\
\end{array}
\]
\caption{Syntax of types.} 
\label{fig:typeSyntax}
\end{figure*}

\subsection{Types}

Figure~\ref{fig:typeSyntax} defines the syntax of \emph{types}.
Behavioral types, ranged over by \(P\), specify the behavior of
statements.  The type $\mathbf{0}$ represents the do-nothing behavior.
The type $P_1;P_2$ is the sequential composition of $P_1$ and $P_2$.
The types $\Malloc$ and $\Free$ represents one allocation and one
deallocation, respectively.  The type $(x)P$ is the behavior that binds
a variable $x$ and behaves as specified by $P$; the type $P$ may be
dependent to $x$.  For example, the behavior of the statement $\LET\ x =
y\ \IN\ \SKIP$ is represented by $(x)\mathbf{0}$ since it binds the
variable $x$ and does nothing. The type $(*x)(P_1,P_2)$ is the behavior
of a branching statement.  This type represents the behavior $P_1$ if
$*x$ is null; $P_2$ otherwise.  The type represents the behavior of a
statement that works as specified in $P$ under the assumption that the
value of $*x$ does not change.  The type $\Endconst$ represents the
behavior that marks the end of the constantness annotation.  The types
$\alpha$ and $\mu\alpha.P$ allows to specify a recursive behavior.  The
type $\mu\alpha.P$ represents the behavior $P$, in which $\alpha$
represents the behavior of $\mu\alpha.P$ itself.  For example,
$\mu\alpha.(\Malloc;\Free;\alpha)$ represents a statement that repeats
an allocation followed by a deallocation forever.  $\Gamma$, a
\emph{type environment}, is a set of variables representing the set of
free variables that may appear in a statement.

The types for procedures, ranged over by $\Psi$, is of the form
$(\vec{x})P$.  This type represents a procedure that takes $\vec{x}$ as
arguments and behaves as specified by $P$, which may be dependent to
$\vec{x}$.  Procedure type environments, ranged over by $\Theta$,
assigns a procedure type to a function name.

% \(k\) represents constant values information, where \(\snull\) represents
% \(\Sirx\) is a null pointer; \(\snnull\) represents \(\Sirx\) is not a
% null pointer; \(\scon\Sirx\) represents \(\Sirx\) should be a constant.

% Constant value environment ranged over by \(F\) is a set of constant
% values information.

% \paragraph{Notation}
% \(filter\_T(F, *x)\) is defined by a pseudcode as follows:
% \[
% \begin{array}{lcl}
%   filter\_T(F, *x) &=& let \; F' = F - \scon\Sirx \; in \\
%   & & if \; \scon\Sirx \notin \; F'\; then \; return \; (F' \backslash \{\snull,\snnull\})\\
%   & & else \; return \; F'
% \end{array}
% \]

Figure~\ref{fig:bdRules} defines the operational semantics of the
behavioral types.  The semantics is defined by a labeled transition
system over configuration of the shape $\langle P, C \rangle$.  Here,
$C$ is the multiset of facts that hold.  We add explanation to
important rules.
\begin{itemize}
 \item \textsc{Tr-Bind}: The behavioral type $(x)P$ picks up a fresh
       name $x'$ and makes transition to $[x'/x]P$.  To understand this
       definition, recall that the rules in
       Figure~\ref{fig:transitionRules} for dealing with statements with
       a binder (i.e., \textsc{Sem-LetNull}, \textsc{Sem-LetEq},
       \textsc{Sem-LetDeref}, and \textsc{Sem-Malloc}) generate a fresh
       variable for the bound variable and add it to the environment $R$
       of the configuration.  The rule \textsc{Tr-Bind} simulates this
       behavior.
 \item \textsc{Tr-Const} and \textsc{Tr-EndConst}: The behavioral type
       $\scon(*x)P$ adds the fact $\scon(*x)$ to $C$ and evolves to
       $P;\Endconst$.  If $\Endconst$ is encountered, then all the facts
       related to $*x$ in $C$ is removed by $\FILTER$.  This is parallel
       to the semantics of $\scon(*x)s$ and $\Endconst$ in
       Figure~\ref{fig:transitionRules}.
 \item \textsc{Tr-ConstNondet1} and \textsc{Tr-ConstNondet2}: If the
       fact $\scon(*x)$ is a member of $C$ and there is no $\NULL(*x)$
       nor $\neg\NULL(*x)$, then the behavioral type $(*x)(P_1,P_2)$
       nondeterministically chooses $P_1$ or $P_2$.  The fact
       $\NULL(*x)$ (resp. $\neg\NULL(*x)$) is added to $C$ if the branch
       $P_1$ (resp. $P_2$) is taken.  This addition of the fact is
       conducted only if $\scon(*x)$ is a member of $C$ (cf.,
       \textsc{Tr-NoConst1} and \textsc{Tr-NoConst2}).
 \item \textsc{Tr-ConstNull} and \textsc{Tr-ConstNotNull}: If
       $\scon(*x)$ is a member of $C$ and if $\NULL(*x)$ (resp.
       $\neg\NULL(*x)$) is a member of $C$, then the behavioral type
       $(*x)(P_1,P_2)$ evolves to $P_1$ (resp. $P_2$).  This happens
       only if (1) the behavioral type $(*x)(P_1,P_2)$ was inside the
       block of certain $\scon(*x)\{\dots\}$ and (2) there was another
       $(*x)(P_1',P_2')$ was in the same block before $(*x)(P_1,P_2)$ is
       reached.  Because the transition of $(*x)(P_1',P_2')$ should have
       registered $\NULL(*x)$ or $\neg\NULL(*x)$ to $C$ and both
       $(*x)(P_1',P_2')$ and $(*x)(P_1,P_2)$ are inside
       $\scon(*x)\{\dots\}$, it is safe to assume that $(*x)(P_1,P_2)$
       takes the same branch as $(*x)(P_1',P_2')$.
 \item \textsc{Tr-NoConst1} and \textsc{Tr-NoConst2}: If $\scon(*x)$ is
       not in the multiset of facts $C$, then the behavioral type
       $(*x)(P_1,P_2)$ nondeterministically evolves to $P_1$ or $P_2$.
\end{itemize}

% Figure~\ref{fig:bdRules} depicts semantics of behavioral types with
% dependent types, and they are given by the labeled transition
% system. The relation \( \langle P, F \rangle \xlongrightarrow{\rho}
% \langle P', F' \rangle \) means that \(P\) can make an action \(\rho\),
% and \(P\) turns into \(P'\) after it makes action \(\rho\); \(F\) and
% \(F'\) record constant value environment before and after making action
% \(\rho\) respectively.


\begin{figure*}
 \begin{minipage}{\textwidth}


\infax[Tr-Skip]
{ \langle \mathbf{0};P, C \rangle \rightarrow \langle P, C \rangle }

\infrule[Tr-Seq]
{ \langle P_1, C \rangle \xlongrightarrow{\rho} \langle P_1', C' \rangle }
{ \langle P_1;P_2, C \rangle \xlongrightarrow{\rho} \langle P_1';P_2, C' \rangle }

%% \begin{minipage}{0.5\textwidth}
%% \infax[Tr-Malloc]
%%  { \langle \Malloc, F \rangle \xlongrightarrow{\Malloc} \langle {\bf 0}, F \rangle }
%% \end{minipage}

\infax[Tr-Malloc]
{ \langle \Malloc, C \rangle \xlongrightarrow{\Malloc} \langle \mathbf{0}, C \rangle }

\infax[Tr-Free]
{ \langle \Free, C \rangle \xlongrightarrow{\Free} \langle \mathbf{0}, C \rangle }

\infrule[Tr-Bind]
{\mbox{$x'$ is fresh.}}
{ \langle (x)P, C \rangle \rightarrow \langle [x'/x]P, C \rangle }

\infax[Tr-Const]
{ \langle \scon\Sirx P, C \rangle \rightarrow \langle P;\Endconst, C + \{\scon\Sirx\} \rangle }

\infrule[Tr-Endconst]
{C' = \FILTER(C, *x)}
{ \langle \Endconst, C \rangle \rightarrow \langle \mathbf{0}, C' \rangle }

\begin{minipage}{0.5\textwidth}
\infrule[Tr-NoConst1]
{ \scon\Sirx \notin C }
{ \langle \Sirx (P_1,P_2), C \rangle \xlongrightarrow{\snull} \langle P_1, C \rangle }
\end{minipage}
\begin{minipage}{0.5\textwidth}
\infrule[Tr-NoConst2]
{ \scon\Sirx \notin C }
{ \langle \Sirx (P_1,P_2), C \rangle \xlongrightarrow{\snnull} \langle P_2, C \rangle }
\end{minipage}
%% \infrule[Tr-NullNotIn]
%% { \snull \notin F \andalso \scon\Sirx \in F }
%% { \langle \Sirx (P_1,P_2), F \rangle \rightarrow \langle P_2, F\cup\{\snnull\} \rangle }

\begin{minipage}{0.5\textwidth}
\infrule[Tr-ConstNull]
{ \snull, \scon\Sirx \in C }
{ \langle \Sirx (P_1,P_2), C \rangle \rightarrow \langle P_1, C \rangle }
\end{minipage}
\begin{minipage}{0.5\textwidth}
\infrule[Tr-ConstNotNull]
{ \snnull, \scon\Sirx \in C }
{ \langle \Sirx (P_1,P_2), C \rangle \rightarrow \langle P_2, C \rangle }
\end{minipage}


\infrule[Tr-ConstNondet1]
{ \snull, \snnull \notin C \andalso \scon\Sirx \in C }
{ \langle \Sirx (P_1,P_2), C \rangle \xlongrightarrow{\snull} \langle P_1, C\cup{\snull} \rangle }

\infrule[Tr-ConstNondet2]
{ \snull, \snnull \notin C \andalso \scon\Sirx \in C }
{ \langle \Sirx (P_1,P_2), C \rangle \xlongrightarrow{\snnull} \langle P_2, C\cup{\snnull} \rangle }

% \begin{minipage}{0.5\textwidth}
% \infax[Tr-ChoiceL]
% { \langle P_1 + P_2, F \rangle \rightarrow \langle P_1, F \rangle }
% \end{minipage}
% \begin{minipage}{0.5\textwidth}
% \infax[Tr-ChoiceR]
% { \langle P_1 + P_2, F \rangle \rightarrow \langle P_2, F \rangle }
% \end{minipage}

\infax[Tr-Rec]
 { \langle \mu\alpha.P, C \rangle \rightarrow \langle
   [\mu\alpha.P/\alpha]P, C \rangle }

% \infax[Tr-LetX*Y]
% { \langle \LET x = *y \; \IN P, F \rangle \rightarrow \langle [x'/x]P, F \rangle }

% \infax[Tr-LetX]
% { \langle \LET x = \NULL \; \IN P, F \rangle \rightarrow \langle [x'/x]P, F \rangle }

%% \infrule[Tr-NNullNotIn]
%% { \snnull \notin F \andalso \scon\Sirx \in F}
%% { \langle \Sirx (P_1,P_2), F \rangle \rightarrow \langle P_1, F\cup\{\snull\} \rangle }
 
%%   \label{df:fv}
%%   \mbox{The set of free variables of behavioral type \(P\) is defined as
%%     follows:}

%%   \( \mathbf{FV}(\bf{0}) = \mathbf{FV}(\Malloc) =
%%   \mathbf{FV}(\Free) = \mathbf{FV}(\alpha) = \emptyset \) 

%%   \( \mathbf{FV}(P_1 + P_2) = \mathbf{FV}(P_1;P_2) = \mathbf{FV}(P_1) \cup \mathbf{FV}(P_2)\)

%%   \( \mathbf{FV}((x)(P_1,P_2)) \) = \{x\} \(\cup \ \mathbf{FV}(P_1) \cup
%%   \mathbf{FV}(P_2)\)
%% \end{myDef}
\end{minipage}
\caption{semantics of behavioral types with dependent types.}
\label{fig:bdRules}
\end{figure*}

\begin{exmp}
Let $P$ be the behavioral type
\[
 \scon(*x)\{(*x)(\Malloc,\mathbf{0});(*x)(\Free,\mathbf{0})\}.
\]
We write $P_1$ for $(*x)(\Malloc,\mathbf{0})$, $P_2$ for
$(*x)(\Free,\mathbf{0})$, $C_1$ for $\set{\scon(*x)}$, $C_2$ for
$\set{\NULL(*x)}$, and $C_3$ for $\set{\neg\NULL(*x)}$.  Then, the
following transition sequence is possible.
\[
 \begin{array}{ll}
                 & \langle P, \emptyset \rangle\\
  \xrightarrow{} & \langle P_1;P_2;\Endconst, C_1 \rangle\\
  \xrightarrow{} & \langle \Malloc;P_2;\Endconst, C_1 \cup C_2 \rangle\\
  \xrightarrow{\Malloc} & \underline{\langle P_2;\Endconst, C_1 \cup C_2 \rangle}\\
  \xrightarrow{} & \langle \Free;\Endconst, C_1 \cup C_2 \rangle\\
  \xrightarrow{\Free} & \langle \Endconst, C_1 \cup C_2 \rangle\\
  \xrightarrow{} & \langle \mathbf{0}, \emptyset \rangle.\\
 \end{array}
\]
Notice that, from the underlined configuration, the transition to
$\langle \mathbf{0};\Endconst, C_1 \cup C_2 \rangle$ is not allowed
because the facts include $\scon(*x)$ and $\NULL(*x)$.  Another possible
transition is
\[
 \begin{array}{ll}
                 & \langle P, \emptyset \rangle\\
  \xrightarrow{} & \langle P_1;P_2;\Endconst, C_1 \rangle\\
  \xrightarrow{} & \langle \mathbf{0};P_2;\Endconst, C_1 \cup C_3 \rangle\\
  \xrightarrow{} & \underline{\langle P_2;\Endconst, C_1 \cup C_3 \rangle}\\
  \xrightarrow{} & \langle \mathbf{0};\Endconst, C_1 \cup C_3 \rangle\\
  \xrightarrow{} & \langle \Endconst, C_1 \cup C_3 \rangle\\
  \xrightarrow{} & \langle \mathbf{0}, \emptyset \rangle.\\
 \end{array}
\]
A transition to $\langle \Free;\Endconst, C_1 \cup C_3 \rangle$ is not
allowed because $C_1 \cup C_3$ contains $\scon(*x)$ and $\neg\NULL(*x)$.
Notice that the path-sensitive behavioral type excludes the possibility
of $\langle P, \emptyset \rangle \xrightarrow{\Malloc} \langle
\mathbf{0}, \emptyset \rangle$ and $\langle P, \emptyset \rangle
\xrightarrow{\Free} \langle \mathbf{0}, \emptyset \rangle$.
\end{exmp}

\subsection{Type judgment}

The type judgment for statements is of the form \(\Theta; \Gamma \vdash
s : P \).  The judgment reads ``the behavior of $s$ is $P$ under the
function type environment $\Theta$ and the variable type environment
$\Gamma$''.

We incorporate subtyping to our type system defined as follows.
\begin{myDef}[Subtyping]
\label{df:subtype} \(C \vdash P_1 \le P_2\) is the largest relation such
that, for any \(P_2'\), $C'$, and \(\rho\), if \( \langle P_2, C \rangle
\xlongrightarrow{\rho} \langle P_2', C' \rangle \), then there exists
\(P_1'\) such that \( \langle P_1, C \rangle \xLongrightarrow{\rho}
\langle P_1', C' \rangle \) and \(C' \vdash P_1' \le P_2'\).  We write
\(P_1 \le P_2\) if \(C \vdash P_1 \le P_2\) for any \(C\).
\end{myDef}

\begin{figure*}
\begin{minipage}{\textwidth}

\begin{minipage}{0.5\textwidth}
\infax[T-Skip]
{\Theta ; \Gamma \vdash \SKIP : \mathbf{0}}
\end{minipage}
\begin{minipage}{0.5\textwidth}
\infrule[T-Seq]
{\Theta ; \Gamma \vdash s_{1} : P_{1} \andalso \Theta ; \Gamma \vdash s_{2} : P_{2}}
{\Theta ; \Gamma \vdash s_{1} ; s_{2} : P_{1};P_{2} }
\end{minipage}


\begin{minipage}{0.5\textwidth}
\infax[T-Assign]
{\Theta ; \Gamma, x, y \vdash *x \leftarrow y : \mathbf{0} }
\end{minipage}
\begin{minipage}{0.5\textwidth}
\infax[T-Free]
{\Theta ; \Gamma, x \vdash \Free(x) : \Free}
\end{minipage}


\begin{minipage}{0.5\textwidth}
\infrule[T-Malloc]
{\Theta ; \Gamma,x \vdash s : P}
{\Theta ; \Gamma \vdash \LET x = \MALLOC\ \IN s \COL \Malloc;(x)P}
\end{minipage}
\begin{minipage}{0.5\textwidth}
\infrule[T-LetEq]
{\Theta ; \Gamma , x, y  \vdash s : P}
{\Theta ; \Gamma, y \vdash \LET x = y\ \IN s : [y/x]P}
\end{minipage}


\begin{minipage}{0.5\textwidth}
\infrule[T-LetDeref]
{\Theta ; \Gamma , x, y  \vdash s : P}
{\Theta ; \Gamma, y \vdash \LET x = *y\ \IN s : (x)P}
\end{minipage}
\begin{minipage}{0.5\textwidth}
\infrule[T-LetNull]
{\Theta ; \Gamma, x  \vdash s : P  \Rtab}
{\Theta ; \Gamma \vdash \LET x = \NULL\ \IN s : (x)P}
\end{minipage}

\infax[T-Endconst] 
{\Theta ; \Gamma,x \vdash \Endconst : \Endconst}

\infrule[T-Const] 
{\Theta ; \Gamma, x \vdash s : P }
{\Theta ; \Gamma,x \vdash \scon\Sirx s : \scon\Sirx P}


%%\begin{minipage}{0.5\textwidth}  \andalso x \in \mathbf{FV}(P_1) \cup \mathbf{FV}(P_2)
\infrule[T-IfNull] 
{\Theta ; \Gamma, x \vdash s_{1} : P_1 \andalso \Theta ; \Gamma, x \vdash s_{2} : P_2}
{\Theta ; \Gamma, x \vdash \IFNULL\Sirx \; \THEN s_{1}\; \ELSE s_{2} : (*x)(P_1,P_2)}
%%\end{minipage}
%%\begin{minipage}{0.5\textwidth}
\infax[T-Call] {\Theta, f \COL (\vec{x})P; \Gamma, \vec{y} \vdash f(\vec{y}) :  [\vec{y}/\vec{x}]P}
%%\end{minipage}

%%\vspace{2mm} 

\infrule[T-Sub]
{\Theta ; \Gamma \vdash s : P_{1} \andalso P_{1} \le P_{2}}
{\Theta ; \Gamma \vdash s : P_{2}}

\infrule[T-Def] { \Theta(f) = (\vec{x})P \andalso \DOM(D) = \DOM(\Theta)
        \andalso \Theta; x_1,\dots,x_n \vdash s \COL P \mbox{ for each
        \(f \mapsto (x_1,\dots,x_n)s \in D\)} } {\vdash D \COL \Theta}
% Program
\infrule[T-Program]
{\vdash D : \Theta \andalso \Theta; \emptyset\vdash s : P}
{\vdash \langle D, s \rangle : P}

\end{minipage}
\caption{Typing rules.}
\label{fig:TypingRules}
\end{figure*}

The type judgment $\Theta; \Gamma \vdash s \COL P$ is defined as the
least relation that satisfies the rules in Figure~\ref{fig:TypingRules}.
We give explanation to several important rules.
\begin{itemize}
 \item \textsc{T-Malloc}: The statement $\LET\ x = \Malloc()\ \IN\ s$
       performs an allocation once, binds the variable $x$, and then
       executes $s$.  The behavior of the whole statement is therefore
       $\Malloc;(x)P$.
 \item \textsc{T-LetEq}: The statement $\LET\ x = y\ \IN\ s$ makes an
       alias $x$ of $y$ and then executes $s$.  Therefore, we rename $x$
       to $y$ in the behavioral type $P$ of $s$.
 \item \textsc{T-LetDeref} and \textsc{T-LetNull}: The statements $\LET\
       x = *y\ \IN s$ and $\LET\ x = \NULL\ \IN\ s$ bind $x$ and then
       execute $s$.  Therefore, the type of the whole statement is
       $(x)P$.  The type system just ignores that what the variable $x$
       is bound to.
 \item \textsc{T-IfNull}: The statement 
       \[
       \IFNULL(*x)\ \THEN\ s_1\ \ELSE\ s_2
       \]
       branches depending on whether $*x$ is null or not.  This behavior
       is abstracted by the type $(*x)(P_1,P_2)$.
 \item \textsc{T-Call}: If the type of the procedure $f$ is
       $(\vec{x})P$, then the type of $f(\vec{y})$ is
       $[\vec{y}/\vec{x}]P$; notice that $P$ may depend on the arguments
       of the procedure.
\end{itemize}

Typing rules for the declarations $D$ and programs $\langle D, s
\rangle$ are also shown in Figure~\ref{fig:TypingRules}.  These rules
are standard.

\subsection{Type soundness}

Although we have not completed the proof concretely, we conjecture that
the following preservation theorem for the judgment $\Theta;\Gamma
\vdash s \COL P$ holds.
\begin{conjecture}[Preservation]
 \label{conj:preservation}
 Suppose that
 \begin{itemize}
  \item $\vdash D \COL \Theta$,
  \item $\Theta;\Gamma \vdash s \COL P$,
  \item $\Gamma \subseteq \DOM(R)$,
  \item $\langle H,R,s,n,C \rangle \xlongrightarrow{\rho} \langle H',R',s',n',C' \rangle$,
 \end{itemize}
 Then, there exist $\Gamma'$ and $P'$ such that
 \begin{itemize}
  \item $\Theta;\Gamma' \vdash s' \COL P'$,
  \item $\Gamma' \subseteq \DOM(R')$,
  \item $\langle P,C \rangle \xLongrightarrow{\rho} \langle P',C' \rangle$.
 \end{itemize}
\end{conjecture}

% \begin{lemma}[Preservation]
% \label{lem:preservation}
% suppose that \( \Theta; \Gamma \vdash \langle H, R, s, n, C \rangle :
% \langle P, F \rangle\). If \( \langle H, R, s, n, C \rangle
% \xlongrightarrow{\rho} \langle H', R', s', n', C' \rangle\) then
% \(\exists P', F'\) s.t. (1) \( \Theta; \Gamma \vdash \langle H', R',
% s', n', C' \rangle : \langle P', F' \rangle\) and (2) \(\langle P, F
% \rangle \xLongrightarrow{\rho} \langle P', F'\rangle \).
% \end{lemma}

The following fact is a corollary of Conjecture~\ref{conj:preservation}.
\begin{conjecture}\label{cor:progAbstract}
 If $\vdash \langle D, s \rangle : P$ and $\langle \emptyset, \emptyset,
 s, n, \emptyset \rangle \xLongrightarrow{\sigma}_D$, then $\langle P,
 \emptyset \rangle \xLongrightarrow{\sigma}$.
\end{conjecture}

\subsection{Verification of memory-leak freedom of a program using 
the inferred behavioral type}

Conjecture~\ref{cor:progAbstract} states that the behavior of a program
is soundly abstracted by its type.  This section describes how the
inferred type can be used to estimate an upper bound of memory
consumption.  We first define ``memory consumption'' of a behavioral
type.

\begin{myDef}
\label{df:okn} We write \(OK_n(P, C)\) if (1) \(\langle P, C \rangle
\xlongrightarrow{\sigma} \langle P', C'\rangle \) implies
\(\sharp_\Malloc(\sigma) - \sharp_\Free(\sigma) \le n\) where
$\sharp_\rho(\sigma)$ is the number of $\rho$ in $\sigma$.
\end{myDef}

\begin{conjecture}
\label{lem:immediateSafety}
If
 \begin{itemize}
  \item $\Theta;\Gamma \vdash s \COL P$,
  \item $\Gamma \subseteq \DOM(R)$,
  \item $\OK_n(P,C)$, and
  \item $n' \ge n$,
 \end{itemize}
 Then, \(\langle H, R, s, n', C \rangle \not\xLongrightarrow{\sigma}^* \OVERFLOW \).
\end{conjecture}

The conjecture above states that, if $\vdash \langle D, s \rangle \COL
P$ and $\OK_n(P,\emptyset)$ hold, then $n$ is an upper bound of the
program.  Therefore, by conducting type inference to the program,
obtaining a type, say, $P$, and estimating an upper bound of $P$, we can
estimate an upper bound of the memory consumption of the program.

% FROM HERE.

% \begin{myDef}
% \label{df:okf} \(OK(F)\) holds if F does not contain both \( \snull \)
% and \( \snnull \).
% \end{myDef}

% \begin{myDef}
%   \label{df:Rdu}
%  we write \( \Theta; \Gamma \vdash \langle H, R, s, n, C \rangle : \langle P,
%   F \rangle\), if \(\Theta; \Gamma \vdash s : P\) and \(OK_n(P, F)\) with \(C \thickapprox F\).  
% \end{myDef}

% The proof is based on the following lemmas: preservation and lack of
% immediate overflow.

% %% \begin{lemma}[Preservation]
% %% \label{lem:preservation}
% %% If \(OK_{n}(P, F)\), \(\Theta; \Gamma \vdash s : P \), \(\vdash D \COL
% %% \Theta\), and \( \langle H, R, s, n, C \rangle \xlongrightarrow{\rho}
% %% \langle H, R', s', n', C' \rangle \), then there exists \(P'\) and \(F'\)
% %% such that (1) \( \Theta; \Gamma^{'} \vdash s' : P'\), (2)
% %% Rab(\(\langle P, F \rangle \xlongrightarrow{\rho} \langle P', F'
% %% \rangle\)), and (3) \(OK_{n'}(P', F')\).
% %% \end{lemma}



% %% \begin{lemma}[Lack of immediate overflow]
% %% \label{lem:immediateSafety}
% %% If \(\Theta; \Gamma \vdash s \COL P \), \(\vdash D \COL \Theta\), and
% %% \(\OK_n(P)\), then \(\langle H, R, s, n \rangle
% %% \not\xlongrightarrow{\Malloc} \OVERFLOW \).
% %% \end{lemma}

% Before showing typing rules for statements in
% Figure~\ref{fig:TypingRules}, we need explain several important
% definitions. The first one is \(OK_n(P, F)\), a predicate, where \(P\)
% represents the behavior of a program which consumes at most \(n\) memory
% cells under constant value environment \(F\).

% \begin{myDef}[\(\sharp_{\rho}(\sigma)\)]
% \label{df:sharf}
% \(\sharp_{\rho}(\sigma)\) is the number of \(\rho\) in the sequence
% \(\sigma\).
% \end{myDef}

% %% \begin{myDef}
% %% \label{df:const}
% %% \(\sconst\) means:\\
% %% \( \forall \sigma',\sigma''\). \(\sigma'\) is a subsequence
% %% of \(\sigma\), and \(\sigma' = \Startconst;\sigma'';\Endconst\) or \(\sigma' = \sigma'';\Endconst\). The \(\sigma''\) does not contain \(\Startconst\) or \(\Endconst \), and it also does not contain both \(\sassx\) and \(\sassxn\).
% %% \end{myDef}

% Intuitively, \(OK_n(P, F)\) represents at very running steps, the
% number of memory cells a program consumed will not exceed the number
% of memory cells the program requires.

% Figure~\ref{fig:TypingRules} shows the typing rules. For example, the
% rule \rn{T-IfNull} represents the behavior of \(\IFNULL\Sirx \; \THEN
% s_{1}\; \ELSE s_{2}\) is abstracted as \(\Sirx(P_1, P_2)\) where
% \(P_1\) and \(P_2\) are the behavior of \(s_1\) and \(s_2\)
% respectively; this conditional statement means that executing \(s_1\)
% if \(\Sirx\) is a null pointer, otherwise \(s_2\).  The typing rule
% \rn{T-Program} represents a program requires at most \(n\) memory
% cells during running under the predication \(OK_n(P, F)\), where \(P\)
% is behavioral type of statement \(s\).


