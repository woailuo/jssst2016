\section{Conclusion}
\label{sec:conclusion}
In order to deal with path-sensitive problem, which results in an
imprecise abstraction even such that verification failed even for a
memory-leak free program, we proposed an extension of the previous
type system with dependent types.  We also described a type
reconstruction algorithm for this extended type system, and we
conducted several experiments to prove whether our idea can deal with
path-sensitivity problem or not.

Our extended type system can deal with path-sensitivity but only for
the guard-part of a conditional is a pointer. Therefore verification
failed on a program where guard-part of a conditional is not a
pointer. Besides, for simplification our extended type system excludes
several features of real-world programs. For example, alias pointers
and variable-sized memory blocks. Encoding the part where a pointer is
a constant one by hand is unrealistic; Our types ignore the size of
the allocated block and our system only counts the number of types
\(\Malloc\) and \(\Free\). Therefore, a program, which contains memory
leaks by allocating huge memory blocks, may seem to be a well-typed
one in our type system.  We need to refine our current type system to
solve these problms.

In order to solve the constraint of form \(\OK_\nu(P, F)\), we fix an
upper bound for \(\nu\) at first, which makes our reconstruction
incomplete. For example, a given program consumes at most \(n\)
numbers of memory cells, but if the \(\nu\) we chose is great than
\(n\), the verification holds; otherwise, if the \(\nu\) is less than
\(n\), the verification failed. The reason is that we have not yet
known whether there exits an \(n\) s.t. \(OK_n(P, F)\).
