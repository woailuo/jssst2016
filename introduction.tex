\section{Introduction}
\label{sec:introduction}

Several programming languages, such as the C language, allow very
flexible memory management by providing \emph{manual memory management
primitives} (e.g. \(\mathtt{malloc}\) and \(\mathtt{free}\) in the C
language).  One can write a program that dynamically allocates and
deallocates a memory cell during execution.

However, manual memory management primitives often cause hard-to-find
bugs.  One can write a program that deallocates a memory cell twice, a
program that forgets to deallocate a memory cell, and a program that
accesses to a memory cell that is already deallocated.  These bugs in
manual memory management often leads to a serious security problem.

Using a static analysis is one of the approaches to detecting such bugs
at the early stage of development.  Much effort has been paid to
powerful and efficient static analysis of a program with manual memory
management~\cite{DBLP:conf/aplas/SuenagaK09,DBLP:conf/pldi/HeineL03,DBLP:conf/sigsoft/XieA05,DBLP:journals/scp/SwamyHMGJ06,DBLP:conf/sas/OrlovichR06,DBLP:conf/issta/SuiYX12}
and development of practical static analyzer such as \emph{Infer}
(\url{http://fbinfer.com/}).

Among the typical bugs in memory usage, we have been investigating a
problem of memory consumption analysis.  We are interested in estimating
an upper-bound of memory consumption of a program with manual memory
management.  As a first step to this goal, in the previous work, we
proposed a \emph{behavioral type system} for such
programs~\cite{tanPPL2014}.

% Therefore, many static verification methods have been proposed to
% guarantee safe memory deallocation. They prove \emph{partial}
% memory-leak freedom: if a program terminates, all the memory cells are
% safe deallocated. As we know that nonterminating programs are very
% common in real-world programmings such as Web servers and operating
% systems. To guarantee \emph{total} memory-leak freedom, if a program
% does not consume unbounded number of memory cells during execution, is a
% very crucial issue.

% \begin{figure*}[t]
% \begin{center}
% \begin{minipage}{0.7\textwidth}
% \begin{verbatim}
%   void h1() {                      void h2() {
%     int *x, *y;                      int *x, *y;
%     x = malloc(sizeof(int));         x = malloc(sizeof(int));
%     y = malloc(sizeof(int));         y = malloc(sizeof(int));
%     free(x); free(y);                h2();
%     h1();                            free(x); free(y);
%   }                                }
% \end{verbatim}
% \end{minipage}\\
% \end{center}
% \caption{Two programs.}
% \label{ex:np}
% \end{figure*}

For example, consider the following C function \verb|h1|.
\begin{verbatim}
  void h1() {
    int *x, *y;
    x = malloc(sizeof(int));
    y = malloc(sizeof(int));
    free(x); free(y);
  }
\end{verbatim}
By using our previous type system, we can infer that this function
behaves according to a \emph{behavioral type}
$\Malloc;\Malloc;\Free;\Free$ that expresses that this function performs
allocation twice and deallocation twice.  By inspecting this behavioral
type, we can obtain information on how much memory this program may
consume at most.  Our type system~\cite{} can also deal with more
complex control statement and recursive function calls.

One problem in the previous type system consists in how we analyze a
branch statement.  Consider the following function \verb|h2|.
\begin{verbatim}
  void h2(int *x) {
    int *y;
    if (x == NULL) { /* (A) */
      y = malloc(sizeof(int));
    }
    if (x == NULL) { /* (B) */
      free(y);
    }
  }
\end{verbatim}
This function contains two sequentially executed branch statements.  Our
previous type system infers $(\Malloc + \TSKIP); (\Free + \TSKIP)$ as a
behavioral type of \verb|h2| ignoring the guard conditions.  The symbol
$+$ in the behavioral type represents a nondeterministic choice.
Therefore, this type expresses that this function may or may not perform
an allocation, and then may or may not perform a deallocation.  We
cannot exclude the possibility that \verb|h2| allocates a memory cell
once without a deallocation from this inferred type.  However, since the
guard \verb|(A)| holds if and only if \verb|(B)|, this program actually
performs an allocation and a deallocation sequentially or does nothing.

The current paper tries to address this problem by introducing a
\emph{path-sensitive behavioral type} $(*x)(P_1, P_2)$.  This type
expresses a behavior $P_1$ if $*x$ is
\verb|NULL| and $P_2$ if $*x$ is not \verb|NULL|.  By using
path-sensitive behavioral types, the type of \verb|h2| can be expressed
by $(*x)(\Malloc, \TSKIP); (*x)(\Free, \TSKIP)$.  With the knowledge
that the value of $*x$ does not change throughout this function, which
is expressed by another constructor of a behavioral type introduced
later, we can conclude that the behavior of \verb|h2| is $(\Malloc;
\Free) + \TSKIP$.

In this paper, we define the syntax and the semantics of the behavioral
types extended with path-sensitive ones, extend the typing rules.  We
also discuss how the type inferred by the type system can be used to
estimate an upper bound of a program.  Although the soundness of the
type system is still a conjecture, we expect our extension is an
important steppingstone for behavior analysis of a program with manual
memory management.

% We also describe how
% to conduct type reconstruction for our type system.  We expect that this
% algorithm leads to a behavioral analyzer for a programming language with
% manual memory management primitives.

% \begin{exmp}\label{ex:ex1}

% Figure~\ref{ex:np} describes partial and total memory-leak freedom.
% Both \(h\) and \(h'\) are partially memory-leak free because they do not
% terminate.  The function \(h\) is totally memory-leak free since it
% consumes at most two cells\footnote{We assume that every memory cell
% allocated by \(\Malloc\) is fixed size to simplify our type system
% introduced in Section~\ref{sec:typesystem}.  Extension with
% variable-length cells is one of our future work.}.  However, the
% function \(h'\), when it is invoked, consumes unbounded number of memory
% cells; hence \(h'\) is not totally memory-leak free.
% \end{exmp}

%%%%

% In order to prove \emph{total} memory deallocation, we proposed a
% behavioral type system in previous study[]. It can abstract the behavior
% of a program by using sequential process whose actions represent manual
% memory management primitives, and our behavior type only consider the
% the number and order of manual memory management primitives and
% recursively calls. For example, the abstract behavioral type of function
% \(h\) in Figure~\ref{ex:np} is
% \(\mu\alpha.\Malloc;\Malloc;\Free;\Free;\alpha\), which represents
% function \(h\) allocates two memory cells, deallocates them, and then
% recursively call itself again; the behavioral type of function \(h'\) is
% abstracted as \(\mu\alpha.\Malloc;\Malloc;\alpha;\Free;\Free\), which
% represents \(h'\) allocates two memory cells, call itself again, and
% then deallocates those two cells. That way we can easily estimate the
% upper bound of memory cells a program consumed.

% Although our previours behavioral type system can abstract the
% behavior of a program and estimate the upper bound of memory
% consumption, verification on abstracted behavioral types are failed in
% some cases. For example, the extracted behavioral type of function
% \(foo\) in Figure~\ref{ex:np2} is \( \mu\alpha.\Malloc;\Malloc;\Malloc
% + \TSKIP;\Free + \TSKIP;\Free;\Free;\alpha \), which expresses that
% function \(foo\) allocates two memory cells, a choice command between
% allocating one memory cell and skipping, a choice command between
% deallocating one cell and skipping, deallocates two cells, and then
% call itself again. Due to the choice behavioral type, the above type
% may be seen as \(
% \mu\alpha.\Malloc;\Malloc;\Malloc;\TSKIP;\Free;\Free;\alpha \), which
% expresses function \(foo\) consumes three memory cells but deallocates
% two memory cells, and then iterates this behavior again. This behavior
% means function \(foo\) consumes unbounded number of memory cells,
% although the original program is \emph{total} memory-leak freedom.

% \begin{exmp}\label{ex:ex2}
%   \begin{figure}[h]
% %% 1  \dtb $foo()$ =  \\
% %% 2    \dtb \dtb  $\LET y = \MALLOC \IN $ \\
% %% 3    \dtb \dtb  $\LET x = \MALLOC \IN $\\
% %% 4    \dtb \dtb  $\IFNULL(*y) \ \THEN \ \SKIP \ \ELSE \LET x_1 = \MALLOC \IN  *x \leftarrow x_1  ;$ \\
% %% 5   \dtb \dtb   $\IFNULL(*y) \ \THEN \ \SKIP \ \ELSE \Free(*x) ;$ \\
% %% 6   \dtb \dtb   $\Free(x)$ ; $\Free(y)$ ; $foo()$
% [error here]
% \caption{a nonterminating program with conditionals}
% \label{ex:np2}
% \end{figure}
% Figure~\ref{ex:np2} describes that functioin \(foo\) is a total
% memory-leak freedom program, because it consumes at most three memory
% cells during execution. This function has two conditionals: if \(*y\)
% is not a null pointer, it will allocates one cell at first conditional
% and deallocates that cell at second conditional, otherwise skips.
% \end{exmp}

% Our current idea is to extend previous behavioral type system with
% dependent types[]. The dependent types takes more precise information
% than traditional types, for example, the type
% \((*x)(\Malloc, \TSKIP)\) is a dependent type, because it dependents
% on the value \((*x)\). See the function \(foo\) again, the current
% behavioral type of it is \(
% \mu\alpha.\Malloc;\Malloc;(*x)(\Malloc, \TSKIP);(*x)(\Free, \TSKIP);\\ \Free;\Free;\alpha
% \). Therefore, the part \((*x)(\Malloc, \TSKIP);\\(*x)(\Free, \TSKIP)\)
% can be seen as \(\Malloc;\Free\) or \(\TSKIP;\TSKIP\) if \((*x)\) does
% not change between these two choices, which we can definitely judge it
% is a total memory-leak free program.

The rest of this paper is structured as
follows. Section~\ref{sec:language} describes an imperative language
with allocation and deallocation primitives and its operational
semantics. Section~\ref{sec:typesystem} introduces the extended
behavioral type system with path-sensitive behavioral
types. Section~\ref{sec:reconstruction} describes the type
reconstruction procedure.
% Section~\ref{sec:experiments} shows some
% experiments and give a discussion;
Section~\ref{sec:relatedwork} discusses the related work.
Section~\ref{sec:conclusion} concludes this paper.
