\section{Related Works}\label{sec:relatedwork}

Memory usage is a crucial issue for real-world program, so lots of
static verification method for memory usage have been
proposed~\cite{DBLP:conf/aplas/SuenagaK09,DBLP:conf/pldi/HeineL03,DBLP:conf/sigsoft/XieA05,DBLP:journals/scp/SwamyHMGJ06,DBLP:conf/sas/OrlovichR06,DBLP:conf/issta/SuiYX12}. However,
these static methods only guaranteed partial memory-leak freedom and
lack of illegal accessing of some pointers like null pointers or
dangling pointers.  In our previous work~\cite{}, we proposed a
behavioral type system, inspired by Kobayashi et
al.~\cite{DBLP:journals/lmcs/KobayashiSW06} which guarantees safety
properties of resources usage for concurrent programs, to estimate the
upper bound number of memory cells a program consumes.  By using our
behavioral type system with other static methods mentioned above, we
can guarantee memory-leak freedom even for nonterminating
programs. But verification failed for path-sensitive programs,
therefore we need to assign more information on behavioral type to
deal with this problem. Our idea is to extend our previous type system
with dependent types.

The dependent type~\cite{DBLP:conf/popl/XiP99,DBLP:conf/pldi/XiP98}
takes more precise information than traditional type, and it can
contain any values in types and appear as arguments and results of
functions~\cite{DBLP:conf/tldi/Norell09}. By using this, we can deal
with the path-sensitive programs, for example, the type
\(\Sirx(\MALLOC, \SKIP)\);\(\Sirx(\FREE, \SKIP)\) depends on value
\(\Sirx(*x)\), if \(*x\) is a null pointer, the type is
\(\SKIP;\SKIP\), otherwise \(\MALLOC;\FREE\).




