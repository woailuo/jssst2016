\section{Related Work}\label{sec:relatedwork}

Behavioral types are widely used in program analysis; a survey of this
area can be found in~\cite{DBLP:journals/csur/HuttelLVCCDMPRT16}.
Several representative uses of the behavioral types are ensuring
correctness of the communications conducted among several
processes~\cite{DBLP:journals/corr/abs-1208-6483,DBLP:conf/popl/HondaYC08,DBLP:journals/tcs/CairesV10,DBLP:journals/tcs/IgarashiK04},
static deadlock-freedom
verification~\cite{DBLP:conf/concur/Kobayashi06,DBLP:journals/acta/Kobayashi05}
and static correctness verification of resource-usage patterns (e.g., a
file hanlder is closed before
termination)~\cite{DBLP:journals/lmcs/KobayashiSW06,DBLP:journals/toplas/IgarashiK05}.
Our current work also can be seen as a static analysis of resource-usage
patterns in which memory is a single and unique resource that is
accessed via the primitives $\Malloc$ and $\Free$.

Static verification of memory-leak freedom has been another interesting
topic of program
verification~\cite{DBLP:conf/aplas/SuenagaK09,DBLP:conf/pldi/HeineL03,DBLP:conf/sigsoft/XieA05,DBLP:journals/scp/SwamyHMGJ06,DBLP:conf/sas/OrlovichR06,DBLP:conf/issta/SuiYX12}.
The main interest of the previous work is guaranteeing the following
property: A pointer to every allocated memory cell will be passed to
$\Free$ eventually.  Our work focuses on the sequence of
$\Malloc$ and $\Free$ that may be conducted by a program, forgetting
about which $\Free$ deallocates which memory cell.

% Memory usage is a crucial issue for real-world program, so lots of
% static verification method for memory usage have been
% proposed~\cite{}. However,
% these static methods only guaranteed partial memory-leak freedom and
% lack of illegal accessing of some pointers like null pointers or
% dangling pointers.  In our previous work~\cite{}, we proposed a
% behavioral type system, inspired by Kobayashi et
% al.~\cite{DBLP:journals/lmcs/KobayashiSW06} which guarantees safety
% properties of resources usage for concurrent programs, to estimate the
% upper bound number of memory cells a program consumes.  By using our
% behavioral type system with other static methods mentioned above, we can
% guarantee memory-leak freedom even for nonterminating programs. But
% verification failed for path-sensitive programs, therefore we need to
% assign more information on behavioral type to deal with this
% problem. Our idea is to extend our previous type system with dependent
% types.

% The dependent type~\cite{DBLP:conf/popl/XiP99,DBLP:conf/pldi/XiP98}
% takes more precise information than traditional type, and it can contain
% any values in types and appear as arguments and results of
% functions~\cite{DBLP:conf/tldi/Norell09}. By using this
